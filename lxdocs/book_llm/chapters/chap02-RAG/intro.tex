\documentclass[../../main.tex]{subfiles}
\begin{document}

\chapter{RAG}

\section{RAG Steps}
	\begin{enumerate}
		\item User Query
		\item Embedding of Query
		\item Semantic Search in Vector DB
		\item Retrieve Relevant Chunks
		\item Augmentation (Query + Chunks)
		\item LLM Generation (Answer)
	\end{enumerate}
	\begin{centerBoxImageStyle}
    {\subfix{images/rag.png}}              % #1 image path
    {1\linewidth}                           % #2 minipage width (box width)
    {0.9\linewidth}                           % #3 image width inside box
    {RAG Steps}                             % #4 caption
    {}                                      % #5 label
\end{centerBoxImageStyle}}
	\clearpage

\section{\textbf{Retrieval-Augmented Generation (RAG)}}
	\begin{enumerate}
		\item A technique that gives large language models (LLMs) additional information (context) along with your prompt.
		
		\item RAG is a type of AI workflow.
		
		\item RAG reduces the need for frequent fine-tuning, but does not completely eliminate it.
		
		\item It sends domain-specific knowledge with prompts, which may not always be present in the training data.
		
		\item \textbf{Retrieval}
		\begin{enumerate}
			\item vector DBs are common but not the only option.
			
			\item Other options : BM25/sparse indexes, hybrid systems, or external search.
		\end{enumerate}

		\item \textbf{Augmentation} : give (prompt+retrieved chunks) to LLM.

		\item \textbf{Generation} : generate response using the Augmented data.
		
		\item RAG as Memory Manager
		\begin{enumerate}
			\item \textbf{Context Window} : Short-term memory (small capacity)
			\item \textbf{Vector DB} : Long-term memory (big capacity)
			\item \textbf{RAG} : Smart connector (brings only what’s needed into context)
		\end{enumerate}
	\end{enumerate}


\section{\textbf{LLM/RAG Design strategy}}

	\begin{enumerate}
		\item \textcolor{red}{Selection of Embedding Model to convert data into embedding vectors.}
		\item \textcolor{red}{Tune parameters of the Embedding Model.}
		\item \textcolor{red}{Tune parameters of Semantic Search.}
		\item Chunking.
		\item Embedding.
		\item Retrieval.
		\item Token prediction — Greedy, Top-\textit{k}, Top-\textit{p}, Temperature.
	\end{enumerate}
	\clearpage

\section{\textbf{Embeddings}}	
	\begin{enumerate}
		\item \textcolor{red}{Are numeric vectors that can have 100s or 1000s of dimensions.}
		\item \textcolor{red}{Embedding models are trained on large datasets.}
		\item \textcolor{red}{Embedding models convert objects into embeddings.}
		\item Embeddings have a close semantic meaning, while there can be some loss.
		\item Embeddings provide dense vectors and enable similarity search.
		\item \textbf{Problems:}
			\begin{itemize}
				\item may miss some content 
				\item may mis‑rank content
				\item bring back results that are close but not exactly right
			\end{itemize}
		\item Quality of embeddings — depends on the model used and how many numbers (dimensions) it creates.
		\item Instead of searching by words, we search using meaning (embedding).
	\end{enumerate}

	\section{Chunking Strategy (Split into Chunks) : determine the size and overlap of each chunk}	
	\begin{enumerate}
		\item Large Document
		\item Decide Chunk Size (small/medium/large)
		\item Decide Overlap (0%, 20%, 30%)
		\item Perform Split 
		\item Store as Embeddings in Vector DB
	\end{enumerate}
	\clearpage

	\section{Context Window}	
	\begin{enumerate}
		\item the memory of the LLM
		\item The upper bound of what the LLM can keep track of
		\item Represents how much context LLM can hold at a given time.
		\begin{simpleTable}{Values and Their meaning}   
    
    \def\cellThreeB{\makecell[l]{Small document.\\Low latency, fast response}}
     
    \begin{tabular}{|c|c|c|} \hline
        LLM Models 		                & Nano, mini, flash		& GPT 4.1, Gemini 2.5 pro 				\\ \hline
        Context window size(tokens)		& 2k to 4k	            & 1 Million		\\ \hline
        Best for			            & \cellThreeB           & Large document				\\ \hline
    \end{tabular}
    
\end{simpleTable}}
	\end{enumerate}

	\section{Context engineering}
	\begin{enumerate}
		\item Similar to prompt engineering, reloading the right context for the question/query.
		\item Engineering for the right context is extremely important as it directly affects the context window, and this is where RAG comes into picture.
		\item RAG manages LLM’s context by retrieving information from a vector DB.
		\item The application can retrieve the right context from the vector DB for the LLM when needed.
		\item Vector DB enables the LLM to store only necessary information instead of the whole document.
		\item Retrieval enables the LLM to add only relevant information in the context window.
	\end{enumerate}

	\section{Strategies to search in a large database of documents.}
	\begin{enumerate}
		\item Attach files to the LLM and query. (there is a limit)
		\item Search the documents based on title/contents. And then rank them based on relevance with the prompt. (time consuming to search entire database of docs)
		\item Summarize all docs and create searchable chunks. In summary there may be losses of the context.
		\item Hence the strategy for RAG search is to convert the data into embedding in Vector DB and perform semantic search. Embedding preserves meaning while there can be some loss
	\end{enumerate}

\end{document}
