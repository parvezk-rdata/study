

\section{ Architecture }
\begin{enumerate}
    \item Python only
    \item FAISS first then Chroma later
    \item OpenAI API first (simpler), later local LLM
    \item LangChain DocumentLoader directly
    \item Start with only .txt loader implemented
    \item i want to expose each module as a python module or as an API.
    \item DocumentLoaderModule
        \begin{enumerate}
            \item Load raw documents from disk and return clean text along with metadata
            \item Provide a standard structure usable by other modules
            \item Directory name: \texttt{document\_loader}
            \item Classes and Methods:
                \begin{enumerate}
                    \item \texttt{BaseDocumentLoader}: \texttt{load\_file(path)}, \texttt{load\_directory(path)}
                    \item \texttt{TextDocumentLoader}: \texttt{load\_file(path)}, \texttt{load\_directory(path)}
                    \item \texttt{PDFDocumentLoader}: \texttt{load\_file(path)}, \texttt{load\_directory(path)}
                    \item \texttt{DocumentLoaderManager}: \texttt{load(path)}
                \end{enumerate}
        \end{enumerate}
    \item TextChunkingModule
        \begin{enumerate}
            \item Split raw documents into smaller overlapping text chunks suitable for embedding generation
            \item Preserve document metadata across chunks for traceability during retrieval
            \item Directory name: \texttt{text\_chunking}
            \item Classes and Methods:
                \begin{enumerate}
                    \item \texttt{BaseTextChunker}  
                    Defines a common interface for all text chunking strategies to ensure consistent behavior across implementations.  
                    Methods: \texttt{chunk\_text(text, chunk\_size, overlap)}, \texttt{chunk\_document(document)}

                    \item \texttt{SimpleTextChunker}  
                    Implements basic text splitting logic using plain Python, suitable for simple experimentation and understanding chunking fundamentals.  
                    Methods: \texttt{chunk\_text(text, chunk\_size, overlap)}, \texttt{chunk\_document(document)}

                    \item \texttt{RecursiveTextChunker}  
                    Uses LangChain’s recursive character-based splitting to intelligently break text while respecting natural language boundaries.  
                    Methods: \texttt{chunk\_text(text, chunk\_size, overlap)}, \texttt{chunk\_document(document)}

                    \item \texttt{TextChunkingManager}  
                    Acts as a unified entry point that applies the selected chunking strategy to a list of documents and returns chunked outputs.  
                    Methods: \texttt{chunk(documents, chunk\_size=1000, overlap=100)}
                \end{enumerate}
        \end{enumerate}
    \clearpage
    \item EmbeddingGenerationModule
        \begin{enumerate}
            \item Generate vector embeddings from text chunks using an embedding model
            \item Abstract embedding provider details to support multiple backends
            \item Ensure embeddings are compatible with vector databases
            \item Directory name: \texttt{embedding\_generation}
            \item Classes and Methods:
                \begin{enumerate}
                    \item \texttt{BaseEmbeddingGenerator}  
                    Defines a standard interface for embedding generation to allow easy switching between embedding providers.  
                    Methods: \texttt{embed\_text(text)}, \texttt{embed\_documents(documents)}

                    \item \texttt{OpenAIEmbeddingGenerator}  
                    Uses OpenAI embedding APIs to generate high-quality vector representations for text chunks.  
                    Methods: \texttt{embed\_text(text)}, \texttt{embed\_documents(documents)}

                    \item \texttt{EmbeddingManager}  
                    Acts as a single access point that applies the selected embedding generator to a list of chunked documents.  
                    Methods: \texttt{generate\_embeddings(documents)}
                \end{enumerate}
        \end{enumerate}
    \item zzz
    \item Data bases : 
\end{enumerate}

