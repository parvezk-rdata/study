

\section{ Architecture }
\begin{enumerate}
    \item Python only
    \item FAISS first then Chroma later
    \item OpenAI API first (simpler), later local LLM
    \item LangChain DocumentLoader directly
    \item Start with only .txt loader implemented
    \item i want to expose each module as a python module or as an API.
    \item DocumentLoaderModule
        \begin{enumerate}
            \item Load raw documents from disk and return clean text along with metadata
            \item Provide a standard structure usable by other modules
            \item Directory name: \texttt{document\_loader}
            \item Classes and Methods:
                \begin{enumerate}
                    \item \texttt{BaseDocumentLoader}: \texttt{load\_file(path)}, \texttt{load\_directory(path)}
                    \item \texttt{TextDocumentLoader}: \texttt{load\_file(path)}, \texttt{load\_directory(path)}
                    \item \texttt{PDFDocumentLoader}: \texttt{load\_file(path)}, \texttt{load\_directory(path)}
                    \item \texttt{DocumentLoaderManager}: \texttt{load(path)}
                \end{enumerate}
        \end{enumerate}
    \item TextChunkingModule
        \begin{enumerate}
            \item Split raw documents into smaller overlapping text chunks suitable for embedding generation
            \item Preserve document metadata across chunks for traceability during retrieval
            \item Directory name: \texttt{text\_chunking}
            \item Classes and Methods:
                \begin{enumerate}
                    \item \texttt{BaseTextChunker}  
                    Defines a common interface for all text chunking strategies to ensure consistent behavior across implementations.  
                    Methods: \texttt{chunk\_text(text, chunk\_size, overlap)}, \texttt{chunk\_document(document)}

                    \item \texttt{SimpleTextChunker}  
                    Implements basic text splitting logic using plain Python, suitable for simple experimentation and understanding chunking fundamentals.  
                    Methods: \texttt{chunk\_text(text, chunk\_size, overlap)}, \texttt{chunk\_document(document)}

                    \item \texttt{RecursiveTextChunker}  
                    Uses LangChain’s recursive character-based splitting to intelligently break text while respecting natural language boundaries.  
                    Methods: \texttt{chunk\_text(text, chunk\_size, overlap)}, \texttt{chunk\_document(document)}

                    \item \texttt{TextChunkingManager}  
                    Acts as a unified entry point that applies the selected chunking strategy to a list of documents and returns chunked outputs.  
                    Methods: \texttt{chunk(documents, chunk\_size=1000, overlap=100)}
                \end{enumerate}
        \end{enumerate}
    \clearpage
    \item EmbeddingGenerationModule
        \begin{enumerate}
            \item Generate vector embeddings for both document chunks and user queries
            \item Abstract embedding provider logic to allow easy switching between APIs and local models
            \item Ensure embedding outputs are compatible with downstream vector databases
            \item Directory name: \texttt{embedding\_generation}
            \item Classes and Methods:
                \begin{enumerate}
                    \item \texttt{BaseEmbeddingGenerator}  
                    Defines a common interface for embedding providers and enforces consistent embedding behavior.  
                    Methods: \texttt{embed\_text(text)}, \texttt{embed\_documents(documents)}

                    \item \texttt{OpenAIEmbeddingGenerator}  
                    Uses OpenAI embedding APIs to generate dense vector representations for both documents and user queries.  
                    Methods: \texttt{embed\_text(text)}, \texttt{embed\_documents(documents)}

                    \item \texttt{EmbeddingManager}  
                    Acts as a unified entry point for embedding generation and routes requests to the selected embedding provider.  
                    Methods: \texttt{embed\_query(text)}, \texttt{embed\_documents(documents)}
                \end{enumerate}
        \end{enumerate}

    \item VectorDatabaseModule
        \begin{enumerate}
            \item Manage vector database operations including indexing, persistence, and similarity search
            \item Store embeddings along with document metadata in a backend-agnostic manner
            \item Abstract vector database implementations to allow easy backend replacement
            \item Directory name: \texttt{vector\_database}
            \item Classes and Methods:
                \begin{enumerate}
                    \item \texttt{BaseVectorDatabase}  
                    Defines a standard interface for all vector database backends to ensure consistent indexing and retrieval behavior.  
                    Methods: \texttt{add\_documents(documents, embeddings)}, \texttt{similarity\_search(query\_embedding, k)}, \texttt{save(path)}, \texttt{load(path)}

                    \item \texttt{FAISSVectorDatabase}  
                    Implements vector storage, persistence, and similarity search using FAISS for efficient nearest-neighbor retrieval.  
                    Methods: \texttt{add\_documents(documents, embeddings)}, \texttt{similarity\_search(query\_embedding, k)}, \texttt{save(path)}, \texttt{load(path)}

                    \item \texttt{VectorDatabaseManager}  
                    Acts as a unified access layer to initialize, query, persist, and restore vector databases during indexing and querying workflows.  
                    Methods: \texttt{index\_documents(documents, embeddings)}, \texttt{search(query\_embedding, k)}, \texttt{persist(path)}, \texttt{restore(path)}
                \end{enumerate}
        \end{enumerate}
    \clearpage

    \item PromptManagementModule
        \begin{enumerate}
            \item Assemble final prompts using user queries and retrieved contextual documents
            \item Encapsulate prompt engineering logic and formatting rules
            \item Enable flexible prompt strategies without modifying LLM or retrieval modules
            \item Directory name: \texttt{prompt\_management}
            \item Classes and Methods:
                \begin{enumerate}
                    \item \texttt{BasePromptBuilder}  
                    Defines a standard interface for building prompts to ensure consistent prompt structure across strategies.  
                    Methods: \texttt{build\_prompt(query, context\_documents)}

                    \item \texttt{RAGPromptBuilder}  
                    Implements a retrieval-augmented prompt by combining user queries with relevant document context.  
                    Methods: \texttt{build\_prompt(query, context\_documents)}

                    \item \texttt{PromptManager}  
                    Acts as a unified access point to select and apply the appropriate prompt-building strategy.  
                    Methods: \texttt{create\_prompt(query, context\_documents)}
                \end{enumerate}
        \end{enumerate}

    \item LLMInteractionModule
        \begin{enumerate}
            \item Interact with large language models using pre-assembled prompts
            \item Abstract LLM provider details to support both API-based and local models
            \item Return generated responses without modifying prompt content
            \item Directory name: \texttt{llm\_interaction}
            \item Classes and Methods:
                \begin{enumerate}
                    \item \texttt{BaseLLMClient}  
                    Defines a standard interface for interacting with language models to ensure backend independence.  
                    Methods: \texttt{generate(prompt)}

                    \item \texttt{OpenAILLMClient}  
                    Implements LLM interaction using OpenAI APIs to generate responses from supplied prompts.  
                    Methods: \texttt{generate(prompt)}

                    \item \texttt{LLMClientManager}  
                    Acts as a unified access layer for selecting and invoking the configured LLM backend.  
                    Methods: \texttt{generate\_response(prompt)}
                \end{enumerate}
        \end{enumerate}
    \item RAGOrchestratorModule
        \begin{enumerate}
            \item Coordinate and manage end-to-end retrieval-augmented generation workflows
            \item Define separate indexing and query execution pipelines
            \item Integrate document loading, chunking, embedding, vector storage, prompt construction, and LLM interaction
            \item Directory name: \texttt{rag\_orchestrator}
            \item Classes and Methods:
                \begin{enumerate}
                    \item \texttt{BaseRAGOrchestrator}  
                    Defines the standard interface for orchestrating indexing and querying workflows.  
                    Methods: \texttt{index\_documents(source\_path)}, \texttt{answer\_query(query)}

                    \item \texttt{RAGOrchestrator}  
                    Implements the complete RAG workflow by coordinating all modular components in the correct sequence.  
                    Methods: \texttt{index\_documents(source\_path)}, \texttt{answer\_query(query)}
                \end{enumerate}
        \end{enumerate}
    \item zzz : 
\end{enumerate}

