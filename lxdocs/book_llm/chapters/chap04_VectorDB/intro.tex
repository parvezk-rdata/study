% \documentclass[12pt,a4paper]{book}
% \documentclass[oneside,openany]{book}

\documentclass[../../main.tex]{subfiles}
\begin{document}
	
\chapter{Vector Data Base} 
	
\section{Vector DB}
	
	\begin{enumerate}
		\item Stores vector embeddings derived from unstructured data 
		\item raw content is linked or stored separately depending on the system
			\begin{simpleTable}{Table with environment}
    \begin{tabular}{|c|c|c|c|} \hline
        Vector & ID & Metadata - date, topic, source & link to the original data \\ \hline
    \end{tabular}
\end{simpleTable}}
		\item Examples - Chroma, Pinecone
		\item designed for vectors only, unlike normal databases
		\item Support for CRUD (Create, Read, Update, Delete)
		\item Allow metadata filtering (e.g., “find only documents from 2024”).
		\item Benefits of Vector DB
		\begin{enumerate}
			\item Flexibility : works for many data types like text, image, audio and video
			\item Scalability : can scale for 100 crores data points.
			\item Speed and performance
		\end{enumerate} 
	\end{enumerate}
	\clearpage

\section{Search in vector DB}
	
	\begin{enumerate}
		\item Semantic search 
		\item Search semantically by meaning not by keyword(exact word overlap)
		\item Measures 
		\begin{enumerate}
			\item How close(similar meaning) the vectors are
			\item Compare vectors using math measures (like angle or distance)
			\item Uses math methods like cosine similarity or inner product instead of exact word matches
		\end{enumerate} 
		\item Use ANN(approximate nearest neighbor) search like HNSW, IVF, PQ
	\end{enumerate}
	\input{\subfix{images/vector_db.tex}}

\section{Retrieval Strategy}
	
	\begin{enumerate}
	\item \textcolor{red}{Scoring and chunk overlap} 
	\item control the threshold of how similar the words should be, use additional filters in the data set
	\item In practice, top‑k with similarity + metadata filters is common
	\item many systems add hybrid retrieval and reranking to boost recall/precision
	\end{enumerate}
	\clearpage
	
\section{Ways to search vectors in Vector Database}
	
	\begin{enumerate}
		\item Exact match 
		\begin{enumerate}
			\item finds a perfect match by matching every vector.
			\item Too slow
		\end{enumerate}
		\item Exact closest Match
		\begin{enumerate}
			\item Finds the true nearest neighbors
			\item Too slow
		\end{enumerate}
		\item Approximate Closest Match (ANN – Approximate Nearest Neighbor)
		\begin{enumerate}
			\item Instead of checking all vectors, it uses indexes
			\item Finds vectors that are likely to be the closest matches
			\item fast, scalable, slightly less accurate
		\end{enumerate}
	\end{enumerate}
	
\section{Vector indexing}
	
	\begin{enumerate}
		\item Why we need it 
		\begin{enumerate}
			\item Can store millions of vectors with many dimensions
			\item Comparing a query vector with all stored vectors will be too slow
		\end{enumerate}
		\item What indexing does
		\begin{enumerate}
			\item creates a smart structure
			\item helps the system quickly find similar vectors without checking every single one
		\end{enumerate}
		\item Trade-off
		\begin{enumerate}
			\item makes the search much faster but can sacrifice some accuracy
			\item it may miss the exact best match sometimes
			\item Instead, it finds vectors that are very close (approximate nearest neighbors)
		\end{enumerate}
		\item Algorithms used
		\begin{enumerate}
			\item Uses special algorithms called Approximate Nearest Neighbor (ANN)
			\item Examples: HNSW, IVF, PQ
		\end{enumerate}
	\end{enumerate}
	\clearpage
	
\section{Vector indexing Approaches : }

	\begin{enumerate}
		\item HNSW  
		\begin{enumerate}
			\item Think of it like a map with layers
			\item Vectors are connected in a graph
			\item At higher layers, only a few broad connections exist
			\item As you move down layers, connections get more detailed
			\item The system can quickly “navigate” through these layers to find similar vectors
		\end{enumerate}
		\item IVF
		\begin{enumerate}
			\item Inverted file index
			\item Divides cluster space into clusters and finds only relevant clusters
			\item When you search, the system looks only in the most relevant clusters, not everywhere
		\end{enumerate}
		
	\end{enumerate}
	
	\clearpage
	
\section{Steps}
	\begin{simpleTable}{Customizing The Cells}
		
    \def\cellOneA{\multirow{3}{*}{1}}
    \def\cellOneB{\multirow{3}{*}{Collect data}}
    \def\cellOneCA{text, audio, images, or videos}
    \def\cellOneCB{note metadata - source ID, timestamp and any special tags}
    \def\cellOneCC{If needed, split into small, meaningful chunks with overlaps}
    
    \def\cellTwoA{\multirow{4}{*}{2}}
    \def\cellTwoB{\multirow{4}{*}{Preprocess}}
    \def\cellTwoCA{Clean the data}
    \def\cellTwoCB{Normalize text (lowercase, remove stopwords if needed)}
    \def\cellTwoCC{Transcribe audio into text, extract frames from video, resize images}
    \def\cellTwoCD{Remove sensitive or private info}
    
    \def\cellThreeA{\multirow{3}{*}{3}}
    \def\cellThreeB{\multirow{3}{*}{Chunking}}
    \def\cellThreeCA{If data is too big, then decide chunk size and overlap}
    \def\cellThreeCB{split into small, meaningful chunks with overlaps}
    
    \def\cellFourA{\multirow{3}{*}{4}}
    \def\cellFourB{\multirow{3}{*}{Choose Tools}}
    \def\cellFourCA{Embedding Models that fits chunk size and for text, images, etc}
    \def\cellFourCB{decide how long the vector is (dimension) }
    \def\cellFourCC{decide how similarity is measured (distance metric)}
    
    \def\cellFiveA{\multirow{3}{*}{5}}
    \def\cellFiveB{\multirow{3}{*}{Create Embeddings}}
    \def\cellFiveCA{Use the model to convert chunks into a vector}
    \def\cellFiveCB{Give each one a fixed ID}
    \def\cellFiveCC{attach metadata for easier searching later (e.g., author, date, topic) }
    
    
    \def\cellSixA{\multirow{3}{*}{6}}
    \def\cellSixB{\multirow{3}{*}{Store in Vector DB}}
    \def\cellSixCA{Save the vector + chunk ID + metadata in the database}
    \def\cellSixCB{database automatically builds index using algorithms (HNSW, PQ) }
    
    
    \def\cellSevenA{\multirow{4}{*}{7}}
    \def\cellSevenB{\multirow{4}{*}{search (query)}}
    \def\cellSevenCA{search text is also turned into a vector}
    \def\cellSevenCB{database finds the closest matches (top-k results)}
    \def\cellSevenCC{You can also add filters (e.g., “only from 2024”)}
    \def\cellSevenCD{Sometimes results are verified with another AI model for better accuracy}
    
    \def\cellEightA{\multirow{3}{*}{8}}
    \def\cellEightB{\multirow{3}{*}{Fetch data}}
    \def\cellEightCA{Use the stored IDs to pull the real text, image, or video}
    \def\cellEightCB{Show it to the user, or pass it to an LLM as extra context in RAG}
    \def\cellEightCC{decide how long to keep things stored or delete them}
    
    
    \begin{tabular}{|c|c|c|} \hline
        \cellOneA       & \cellOneB     &   \cellOneCA      \\ \cline{3-3}
                        &				&   \cellOneCB     	\\ \cline{3-3}
                        &				&   \cellOneCC      \\ \hline
        \cellTwoA       & \cellTwoB     &   \cellTwoCA      \\ \cline{3-3}
                        &				&   \cellTwoCB     	\\ \cline{3-3}
                        &				&   \cellTwoCC      \\ \cline{3-3}
                        &				&   \cellTwoCD      \\ \hline
        \cellThreeA     & \cellThreeB   &   \cellThreeCA    \\ \cline{3-3}
                        &				&   \cellThreeCB    \\ \hline
        \cellFourA      & \cellFourB    &   \cellFourCA     \\ \cline{3-3}
                        &				&   \cellFourCB    	\\ \cline{3-3}
                        &				&   \cellFourCC     \\ \hline
        \cellFiveA      & \cellFiveB    &   \cellFiveCA     \\ \cline{3-3}
                        &				&   \cellFiveCB     \\ \cline{3-3}
                        &				&   \cellFiveCC     \\ \hline
        \cellSixA       & \cellSixB     &   \cellSixCA      \\ \cline{3-3}
                        &				&	\cellSixCB      \\ \hline
        \cellSevenA     & \cellSevenB   &   \cellSevenCA    \\ \cline{3-3}
                        &				&   \cellSevenCB    \\ \cline{3-3}
                        &				&   \cellSevenCC    \\ \cline{3-3}
                        &				&   \cellSevenCD    \\ \hline
        \cellEightA     & \cellEightB   &   \cellEightCA    \\ \cline{3-3}
                        &				&   \cellEightCB    \\ \cline{3-3}
                        &				&   \cellEightCC    \\ \hline
    
    \end{tabular}
\end{simpleTable}}

	
\section{ \textcolor{red}{Open source technologies for vector databases} }
	
\clearpage

\end{document}