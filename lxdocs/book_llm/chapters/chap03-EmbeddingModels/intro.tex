\documentclass[../../main.tex]{subfiles}
\begin{document}

\chapter{Embedding Models}

% \input{\subfix{images/Embedding_Models.tex}}
%  \begin{simpleTable}{Values and Their meaning}   
    
    \def\cellThreeB{\makecell[l]{Small document.\\Low latency, fast response}}
     
    \begin{tabular}{|c|c|c|} \hline
        LLM Models 		                & Nano, mini, flash		& GPT 4.1, Gemini 2.5 pro 				\\ \hline
        Context window size(tokens)		& 2k to 4k	            & 1 Million		\\ \hline
        Best for			            & \cellThreeB           & Large document				\\ \hline
    \end{tabular}
    
\end{simpleTable}}

\section{Embeddings Vectors}
	\begin{enumerate}
		\item Are numeric vectors with many dimensions.
		\item often have 100 to 2,000+ dimensions.
		\item Vectors are typically dense (many non-zero values)
		\item Similarity check
			\begin{enumerate}
				\item We compare them , and find how “close”  two vectors are.
				\item Common methods: cosine similarity or Euclidean distance
				\item Help grouping related ideas even if words differ. Example, “car” and “automobile” will have similar embeddings
			\end{enumerate}
		\item Problems
			\begin{enumerate}
				\item Embeddings have a close semantic meaning, while there can be some loss.
				\item may miss some content
				\item may mis‑rank content : similar‑but‑wrong items can appear above correct ones
				\item bring back results that are close but not exactly right
			\end{enumerate}
		\item Instead of searching by words we search using meaning(embedding)
	\end{enumerate}
	\clearpage

	\begin{simpleTable}{}
    
    \def\cellOneA{Modality}
    \def\cellOneB{Type of data - Text, image, audio, video, or a mix}

    \def\cellTwoA{Chunking size}
    \def\cellTwoB{Some models handle longer chunks better than others}

    \def\cellThreeA{\multirow{3}{*}{Embedding size}}
    \def\cellThreeBA{Number of dimensions in vector}
    \def\cellThreeBB{Big vectors slower, memory intensive, capture more meaning}
    \def\cellThreeBC{Small vectors faster and lighter, but may lose detail}
    
    \def\cellFourA{\multirow{2}{*}{Latency(speed) and cost}}
    \def\cellFourBA{Less accurate and quick}
    \def\cellFourBB{slower but more accurate}
    
    \def\cellFiveA{\multirow{2}{*}{Domain or specialty}}
    \def\cellFiveBA{Domain like medical or legal}
    \def\cellFiveBB{Domain-specific models usually perform better for specialized text}
    
    \def\cellSixA{\multirow{2}{*}{Keyword vs meaning focus}}
    \def\cellSixBA{If exact word matches are important, combine sparse + dense models}
    \def\cellSixBB{If meaning is enough, dense embeddings alone work fine}

    \def\cellSevenA{Languages}
    \def\cellSevenB{multilingual model to handle multiple languages}
    
    \def\cellEightA{\multirow{2}{*}{Evaluation}}
    \def\cellEightBA{test on a small sample of your own data}
    \def\cellEightBB{Chose model which gives the best balance of accuracy, speed, and cost}

    \begin{tabular}{|c|c|} \hline
        \cellOneA       & \cellOneB         \\ \hline
        \cellTwoA       & \cellTwoB         \\ \hline
        \cellThreeA     & \cellThreeBA      \\ \cline{2-2}
                        & \cellThreeBB      \\ \cline{2-2}
                        & \cellThreeBC      \\ \hline
        \cellFourA      & \cellFourBA       \\ \cline{2-2}
                        & \cellFourBB       \\ \hline
        \cellFiveA      & \cellFiveBA       \\ \cline{2-2}
                        & \cellFiveBB       \\ \hline
        \cellSixA       & \cellSixBA        \\ \cline{2-2}
                        & \cellSixBB        \\ \hline
        \cellSevenA     & \cellSevenB       \\ \hline
        \cellEightA     & \cellEightBA      \\ \cline{2-2}
                        & \cellEightBB      \\ \hline
    \end{tabular}
\end{simpleTable}}

\section{Embedding Models(Maps)}
	
	\begin{enumerate}
		\item Are special AI models.
		\item Embedding models (\textit{map}) convert data (objects) into embeddings.
		\item Embedding models are trained on large data sets and learn to turn data into numbers.
		\item Types of data include: text, image, audio, video, code, graphs etc.
		\item Examples:
		\begin{enumerate}
			\item Image to text: \textbf{CLIP}
			\item Text: \textbf{GloVe} (old), \textbf{Sentence-BERT}, \textbf{BGE}
			\item Speech/Audio: \textbf{Wav2Vec}
		\end{enumerate}
	\end{enumerate}

	\section{ Why need vectors}
		\begin{enumerate}
			\item Most ML models can’t use raw text directly
			\item they need the text converted into numbers first.
		\end{enumerate}
	\clearpage

	\section{Choice of Embedding Model depends on various factors}
	\begin{simpleTable}{}
    
    \def\cellOneA{Modality}
    \def\cellOneB{Type of data - Text, image, audio, video, or a mix}

    \def\cellTwoA{Chunking size}
    \def\cellTwoB{Some models handle longer chunks better than others}

    \def\cellThreeA{\multirow{3}{*}{Embedding size}}
    \def\cellThreeBA{Number of dimensions in vector}
    \def\cellThreeBB{Big vectors slower, memory intensive, capture more meaning}
    \def\cellThreeBC{Small vectors faster and lighter, but may lose detail}
    
    \def\cellFourA{\multirow{2}{*}{Latency(speed) and cost}}
    \def\cellFourBA{Less accurate and quick}
    \def\cellFourBB{slower but more accurate}
    
    \def\cellFiveA{\multirow{2}{*}{Domain or specialty}}
    \def\cellFiveBA{Domain like medical or legal}
    \def\cellFiveBB{Domain-specific models usually perform better for specialized text}
    
    \def\cellSixA{\multirow{2}{*}{Keyword vs meaning focus}}
    \def\cellSixBA{If exact word matches are important, combine sparse + dense models}
    \def\cellSixBB{If meaning is enough, dense embeddings alone work fine}

    \def\cellSevenA{Languages}
    \def\cellSevenB{multilingual model to handle multiple languages}
    
    \def\cellEightA{\multirow{2}{*}{Evaluation}}
    \def\cellEightBA{test on a small sample of your own data}
    \def\cellEightBB{Chose model which gives the best balance of accuracy, speed, and cost}

    \begin{tabular}{|c|c|} \hline
        \cellOneA       & \cellOneB         \\ \hline
        \cellTwoA       & \cellTwoB         \\ \hline
        \cellThreeA     & \cellThreeBA      \\ \cline{2-2}
                        & \cellThreeBB      \\ \cline{2-2}
                        & \cellThreeBC      \\ \hline
        \cellFourA      & \cellFourBA       \\ \cline{2-2}
                        & \cellFourBB       \\ \hline
        \cellFiveA      & \cellFiveBA       \\ \cline{2-2}
                        & \cellFiveBB       \\ \hline
        \cellSixA       & \cellSixBA        \\ \cline{2-2}
                        & \cellSixBB        \\ \hline
        \cellSevenA     & \cellSevenB       \\ \hline
        \cellEightA     & \cellEightBA      \\ \cline{2-2}
                        & \cellEightBB      \\ \hline
    \end{tabular}
\end{simpleTable}}
	\clearpage

	\section{Embedding Models Details}
	\begin{enumerate}
		\item \textbf{Word Embedding Models} are trained on huge collections of text (like Wikipedia, books, or articles).
	
		\item \textbf{Preprocessing before training:}
		\begin{enumerate}
			\item \textbf{Tokenization:} Splitting text into words or pieces.
			\item \textbf{Removing stopwords and punctuation:} Sometimes done depending on the method.
		\end{enumerate}
	
		\item \textbf{Context Window:}
		\begin{enumerate}
			\item The model looks at a sliding window of words.
			\item \textbf{Example:} In the sentence ``The cat sat on the mat’’, if the target word is ``cat'', its context might be [The, sat].
			\item This helps the model learn how words appear together.
		\end{enumerate}
	
		\item \textbf{Prediction Task:}
		\begin{enumerate}
			\item The model is trained to predict words from context (or vice versa).
			\item \textbf{Example methods:}
			\begin{enumerate}
				\item \textbf{CBOW (Continuous Bag of Words):} Predict target word from context.
				\item \textbf{Skip-gram:} Predict context words from a target word.
			\end{enumerate}
		\end{enumerate}
	
		\item \textbf{Learning Word Relationships:}
		\begin{enumerate}
			\item During training, the model adjusts its internal parameters.
			\item Words that appear in similar contexts end up with similar vectors.
			\item \textbf{Example:} “king” and “queen” are closer to each other than “king” and “banana”.
		\end{enumerate}
	\end{enumerate}
	\clearpage

	\section{NLP}
	\begin{enumerate}
		\item helps machines understand human language
		\item NLP tasks that use word embeddings
	\end{enumerate}

	\begin{simpleTable}{}
    
    \def\cellOneA{Text Classification}
    \def\cellOneB{Spam/non-spam,  news topic categorization, detect patterns}

    \def\cellThreeA{\multirow{2}{*}{Named Entity Recognition NER}}
    \def\cellThreeBA{Finds and classify entities like people, places, organization}
    \def\cellThreeBB{Help understand context. Apple can mean fruit or company}
    
    \def\cellFourA{Word Similarity}
    \def\cellFourB{Measures closeness of two words (e.g., happy $\approx$ joyful)}
    
    \def\cellFiveA{\multirow{2}{*}{Word Analogy}}
    \def\cellFiveBA{capture relationships between words as directions in vector space}
    \def\cellFiveBB{ex - (king : man) = (queen : woman)}
    
    \def\cellSixA{\multirow{3}{*}{Document Similarity}}
    \def\cellSixBA{Clustering related articles}
    \def\cellSixBB{Finding similar documents}
    \def\cellSixBC{Recommending similar items (e.g., products, articles, movies).}

    \begin{tabular}{|c|c|} \hline
        \cellOneA       & \cellOneB         \\ \hline
        \cellThreeA     & \cellThreeBA      \\ \cline{2-2}
                        & \cellThreeBB      \\ \hline
        \cellFourA      & \cellFourB        \\ \hline
        \cellFiveA      & \cellFiveBA       \\ \cline{2-2}
                        & \cellFiveBB       \\ \hline
        \cellSixA       & \cellSixBA        \\ \cline{2-2}
                        & \cellSixBB        \\ \cline{2-2}
                        & \cellSixBC        \\ \hline
    \end{tabular}
\end{simpleTable}}
	
	\section{Word Embedings}
	\begin{enumerate}
		\item {\color{red}Are created by training models on large corpus of text like Wikipedia. Preprocessing, tokenization, removing stop words and punctuators.}
		\item {\color{red}A sliding context window identifies target and context words which allow the model to learn word relationships.}
		\item {\color{red}The model is trained to predict based on their context and it positions semantically similar words close together in vector space.}
		\item {\color{red}Throughout the training the parameters are adjusted.}
	\end{enumerate}

	
	\section{Embedding types}

	\begin{enumerate}
		\color{red}
		\item \textbf{Frequency Based:} uses frequency of words in a text.
		\begin{enumerate}
			\item \textbf{TF-IDF:} Term Frequency–Inverse Document Frequency. Words frequent in one document but not frequent in others. Common words like \textit{the}, \textit{and}, \textit{or} etc.\ have lower TF-IDF scores.
		\end{enumerate}
		\item \textbf{Prediction Based Embedding:} captures semantic relationships and contextual information between words. For example, the word \textit{dog} is related to words like \textit{bark}, \textit{bite}, \textit{tail}, etc.
	\end{enumerate}

	\clearpage

	\section{Models to generate embeddings}

	\begin{enumerate}
		\item \textcolor{red}{Word2Vec (Google)}
			\begin{enumerate}
				\item \textcolor{red}{CBOW (Continuous Bag of Words): predicts a target word based on its surrounding words.}
				\item \textcolor{red}{Skip-gram: predicts context words given a target word.}
			\end{enumerate}
		\item \textcolor{red}{GloVe}
			\begin{enumerate}
				\item \textcolor{red}{Global Vectors for Word Representation}
				\item \textcolor{red}{Uses co-occurrence statistics to create a vector space.}
				\item \textcolor{red}{Analyzes how often words appear together in the entire corpus (collection of text).}
			\end{enumerate}
	\end{enumerate}

	\section{Transformer models : }
	\begin{enumerate}
		\item \textcolor{red}{Traditional models use a fixed vector for each word.}
		\item \textcolor{red}{Uses contextual based embedding.}
		\item \textcolor{red}{Representation of a word changes based on surrounding context.}
		\item \textcolor{red}{Example: the word \textbf{bank} has two different meanings in the following sentences:}
		\begin{enumerate}
			\item \textcolor{red}{I go to the bank to deposit money.}
			\item \textcolor{red}{I sit on the bank of a river.}
		\end{enumerate}
	\end{enumerate}
	
\end{document}