\section{Summerize abstract for me in easy language for a novice in very easy sentences. }

\begin{enumerate}
  \item LLMs can read and understand large amounts of study material. This makes them useful for creating quizzes, summarizing lessons, and helping teachers plan their classes
  \item A major concern is the hallucination problem, where models generate factually incorrect or misleading information due to their probabilistic nature (Z. Li et al., 2025; Ji, Lee, et al., 2023; Ji, Liu, et al., 2023). 
  \item Another limitation of LLM is the static knowledge embedded within LLMs, which makes them unable to reflect the latest curricular updates or scientific advancements (Mallen et al., 2023; Meng et al., 2022; Zhang et al., 2024). To address the limitations, researchers have introduced RAG.
  \item Different from standard LLMs, which rely solely on their pre-trained knowledge, RAG first retrieves relevant documents from an external knowledge base before generating responses. This approach improves factual accuracy, knowledge freshness, and transparency, making it particularly suitable for educational applications where precise and verifiable information is crucial. By integrating retrieval-based knowledge augmentation, RAG systems can provide students with explanations, adapt to curriculum changes, and mitigate the risks of misinformation
\end{enumerate}
