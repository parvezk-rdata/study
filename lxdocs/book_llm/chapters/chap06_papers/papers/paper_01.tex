\documentclass[../../../main.tex]{subfiles}
\begin{document}


% \input{\subfix{images/Embedding_Models.tex}}
% \begin{simpleTable}{Values and Their meaning}   
    
    \def\cellThreeB{\makecell[l]{Small document.\\Low latency, fast response}}
     
    \begin{tabular}{|c|c|c|} \hline
        LLM Models 		                & Nano, mini, flash		& GPT 4.1, Gemini 2.5 pro 				\\ \hline
        Context window size(tokens)		& 2k to 4k	            & 1 Million		\\ \hline
        Best for			            & \cellThreeB           & Large document				\\ \hline
    \end{tabular}
    
\end{simpleTable}}

\chapter{MCQGen: A Large Language Model-Driven MCQ Generator for Personalized Learning} 
	
	\section{Information}
	
	\begin{enumerate}
		\item IEEE Access
		\item DOI : 10.1109/ACCESS.2024.3420709
		\item \url{https://drive.google.com/file/d/1KkfsKJhbx3S2hLeA6iMZsXpkQdwaRPMb/view?usp=drive_link}
	\end{enumerate}
	
    \begin{verbatim*}
        tasks.json for windows miktex
        "group": {
            "kind": "build",
            "isDefault": true
          }
      \end{verbatim*}

      \section{Ideas}
	\begin{enumerate}
    \item Convert a text/paragraph into a fill in the blank question. Identify places which should be converted to blanks. Example To help student memorize a concept or story. 
    \item create a learning model(html component based) to generate learning content and render test say fill in the blank, mcq, short answer, etc. Which suggest areas to focus more and weakness.
    \item Generate possible confusions for a concept/text. Use feedback to determine confusions. Example for a question student might be confused and select non correct option. These confusions should determine area to focus. 
    \item For a give text/topic/content, extract/generate goals(learning paths) to reach expert level. Identify stages/goals and logically arrange them to reach expert level starting from novice. Example you should know A then B then C and so on till T to get full understanding of a topic.
  \end{enumerate}

	\section{TL;DR of this paper aimed at a novice school student(beginner-friendly) in point vise manner}
	
	\begin{enumerate} 
    \item Teachers spend a lot of time making good multiple-choice questions (MCQs) for students. 
    \item MCQGen is a smart computer program that uses a big language model to help make these questions quickly. 
    \item It carefully picks information from textbooks or notes to make sure questions are correct and meaningful. 
    \item The program makes sure each question has one right answer and several tricky but wrong choices. 
    \item MCQGen can change question difficulty based on how well the student learns. 
    \item This helps students practice better by giving questions that fit their learning needs. 
    \item The system also double-checks questions so they don’t have mistakes or confusing wording. 
    \item Overall, MCQGen helps teachers save time and helps students learn smarter using personalized questions. 
  \end{enumerate}

  \section{Summarize this paper in aimed at a beginner school student. Cover these aspects: the problem the paper addresses, why this problem matters, what the authors did to solve it, how they tested or evaluated their work, and the main result or finding.}
  \begin{itemize} 
    \item \textbf{Problem:} Teachers need to create many good multiple-choice questions (MCQs) for students, but making these questions takes a lot of time and skill. 
    \item \textbf{Why it matters:} Having many high-quality questions helps students practice better and learn more effectively, but creating those questions manually is hard and slow. 
    \item \textbf{What the authors did:} The authors made a computer system called MCQGen that uses a large language model (LLM) to automatically create MCQs. It chooses relevant content, creates a question and correct answer, makes tricky wrong choices, and adjusts the question difficulty to fit the student’s level. 
    \item \textbf{How they tested it:} They ran experiments comparing MCQGen to older methods by checking if the questions were correct, clear, and matched the lesson. They also tested if the questions suited different student needs. 
    \item \textbf{Main result:} MCQGen made better questions that were more accurate, had good wrong answers, and could be personalized to help students learn better than previous automatic methods.
  \end{itemize}

	\section{List all glossary terms from this paper and explain each, aimed at a beginner school student.}
	  \begin{itemize} 
      \item \textbf{Multiple-Choice Question (MCQ):} A question with one correct answer and several wrong answers for you to choose from. 
      \item \textbf{Large Language Model (LLM):} A very smart computer program that can understand and write human language like stories, answers, or questions. 
      \item \textbf{Distractor:} A wrong answer in a multiple-choice question that looks believable to make the question challenging. 
      \item \textbf{Personalized Learning:} Teaching that changes the questions or lessons to fit what each student needs or knows best. 
      \item \textbf{Question Stem:} The main part of a question that tells you what is being asked. 
      \item \textbf{Answer Key:} The correct answer to a question. 
      \item \textbf{Hallucination (in AI):} When the computer makes up wrong or made-up information instead of facts. \item \textbf{Pipeline:} A step-by-step process where each step helps make sure the final result (like a good question) is correct and useful. 
      \item \textbf{Validation:} Checking carefully if something is correct, such as checking if the questions and answers make sense. 
      \item \textbf{Difficulty Level:} How easy or hard a question is for a student. 
    \end{itemize}

    \section{What new thing the paper as done or proposed and what was the earlier approach , aimed at a beginner school student.}
      \begin{itemize} 
        \item \textbf{What was done before:} 
          \begin{itemize} 
            \item Earlier computer programs made multiple-choice questions (MCQs) using simple rules or small models. 
            \item These old methods often created questions that were easy, not very clear, or had bad wrong answers that were either too obvious or confusing. 
            \item Sometimes, the questions didn’t match the lesson well, and personalized learning was not common. 
          \end{itemize} 
          \item \textbf{What this paper did new:} 
          \begin{itemize} 
            \item The paper introduced MCQGen, which uses a big smart model called a Large Language Model (LLM) to make questions. 
            \item MCQGen carefully picks content from lessons to make sure questions are correct and related to the topic. 
            \item It makes tricky but fair wrong answers (called distractors) so students think harder. 
            \item It can change question difficulty and topics based on what the student needs, making learning personalized. 
            \item The paper also has a multi-step checking system that filters out bad or confusing questions automatically. 
          \end{itemize} 
          \item \textbf{Why this is better:} 
            \begin{itemize} 
              \item MCQGen makes questions faster and better than old programs. 
              \item Students get questions that match their learning level and needs. 
              \item Teachers save time and students learn smarter. 
          \end{itemize} 
    \end{itemize}
  

\end{document}