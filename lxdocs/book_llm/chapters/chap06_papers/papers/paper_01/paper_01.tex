\documentclass[../../../../main.tex]{subfiles}
\begin{document}


% \input{\subfix{images/Embedding_Models.tex}}
% \begin{simpleTable}{Values and Their meaning}   
    
    \def\cellThreeB{\makecell[l]{Small document.\\Low latency, fast response}}
     
    \begin{tabular}{|c|c|c|} \hline
        LLM Models 		                & Nano, mini, flash		& GPT 4.1, Gemini 2.5 pro 				\\ \hline
        Context window size(tokens)		& 2k to 4k	            & 1 Million		\\ \hline
        Best for			            & \cellThreeB           & Large document				\\ \hline
    \end{tabular}
    
\end{simpleTable}}

\chapter{MCQGen: A Large Language Model-Driven MCQ Generator for Personalized Learning} 
	
	\section{Information }
	
	\begin{enumerate}
		\item \cite{Hang2024}
    \item IEEE Access , DOI : 10.1109/ACCESS.2024.3420709
		\item \url{https://drive.google.com/file/d/1KkfsKJhbx3S2hLeA6iMZsXpkQdwaRPMb/view?usp=drive_link}
	\end{enumerate}

  \clearpage
  \section{Pipeline stages which convert input to final output.}
  \begin{enumerate}
    \item \textbf{Knowledge Base Preparation (Preparing an indexed vector DB to act as Knowledge Base)}
      \begin{enumerate}
          \item Stage-1: Regex-Based Character Filtering and Normalization
          
          \item Stage-2: Text Encoding Standardization
          
          \item Stage-3: Text Tokenization
          
          \item Stage-4: Text Chunking/Segmentation
          
          \item Stage-5: Embedding/Vectorization
          
          \item Stage-6: Vector Database Indexing
      \end{enumerate}
    
    \item \textbf{Query Processing and Retrieval}
      \begin{enumerate}
          \item Stage-7: Query Formulation
          
          \item Stage-8: Semantic Retrieval of chunks from vector DB
      \end{enumerate}
    
    \item \textbf{MCQ Generation and Refinement}
    \begin{enumerate}
        \item Stage-9:  Construction of Custom Combined Prompt
        \item Stage-10: MCQ Generation (Initial draft and Context-Aware[chunk based] )
        \item Stage-11: Prompt Construction for Chain-of-Thought
        \item Stage-12: MCQ Generation(refined using Chain-of-Thought Guided)
    \end{enumerate}
    
    \item \textbf{Self-Refinement Loop}
    \begin{enumerate}
        \item Stage-13: Prompt construction for LLM to generate scores and instructions
        \item Stage-14: LLM generates scores and instructions (critique)
        \item Stage-15: Prompt construction for LLM to refine the MCQs
        \item Stage-16: LLM generates refined MCQs (revised stem, key unchanged)
    \end{enumerate}

    
    \item \textbf{Quality Assurance and Output}
    \begin{enumerate}
        \item Stage-17: MCQ Validation and Evaluation
        
        \item Stage-18: Final Output Formatting
    \end{enumerate}
  \end{enumerate}}

  \clearpage
  \noindent

\tikzstyle{startstop} = [rectangle, rounded corners, 
minimum width=3cm, 
minimum height=1cm,
text centered, 
draw=black, 
fill=red!30]

\tikzstyle{io} = [trapezium, 
trapezium stretches=true, % A later addition
trapezium left angle=70, 
trapezium right angle=110, 
minimum width=3cm, 
minimum height=1cm, text centered, 
draw=black, fill=blue!30]

\tikzstyle{process} = [rectangle, 
    minimum width=3cm, 
    minimum height=1cm, 
    text centered, 
    text width=3cm, 
    text=white,
    draw=black, 
    fill={rgb,255:red,146; green,168; blue,209}  
    % fill=orange!30
]

\tikzstyle{decision} = [diamond, 
minimum width=3cm, 
minimum height=1cm, 
text centered, 
draw=black, 
fill=green!30]

\tikzstyle{db} = [cylinder, 
    draw = violet, 
    text = purple,
    cylinder uses custom fill, 
    cylinder body fill = magenta!10, 
    cylinder end fill = magenta!40,
    minimum height=2cm,
    aspect = 0.2, 
    shape border rotate = 90, 
    text centered]

\tikzstyle{cld} = [cloud,
	draw =blue,
	text=cyan,
	% fill = gray!10,
	% minimum width = 3cm,
	% minimum height = 2cm, 
    text centered]

\tikzstyle{arrow} = [thick,->,>=stealth]




%%%%%%%%%%%%%%%%%%%%%%%%%%%%%%%%%%%%%%%%%%%%%%%%%%%%%%%%%%%%%%%%%%%%%%%%%%%%


%%%%%%%%%%%%%%%%%%%%%%%%%%%%%%%%%%%%%%%%%%%%%%%%%%%%%%%%%%%%

\begin{tikzpicture}[node distance=1.5cm]

% Nodes with explicit positioning to prevent overlap
\node[process] (ui) {User input};
\node[process, right=of ui] (emb) {Embeddings};
\node[db, text width=1.8cm, right=of emb] (db) {Vector\\Database};
\node[cld, text width=1.8cm, right=of db] (chunks) {Relevant\\Chunks};

% \node[process, right=of prompt] (rag) {RAG System};
% \node[db, text width=1.8cm, right=of rag] (db) {Vector\\Database};
% \node[startstop, below=3cm of rag] (chunks) {Relevant\\Chunks};
% \node[startstop, right=of rag] (chunks) {Relevant\\Chunks};

% Arrows
\draw[arrow] (ui) -- (emb);
\draw[arrow] (emb) -- (db);
\draw[arrow] (db) -- (chunks);

\end{tikzpicture}}
  \vspace{0.5cm}
  \noindent\begin{tikzpicture}[
	node distance=2.2cm,
	scale=0.95,
	transform shape,
	framed,
	background rectangle/.style={draw=black, thick}
]

% Nodes
\node[rectangle_a_blue] (browser) {Web \\ Browser};

\node[rec_c_green, right=3cm of browser] (vscode) {VS Code Server\\(on AWS EC2)};

\node[rec_b_pink, right=3cm of vscode] (app) {User \\ Application \\(LLM Client)};

\node[cylinder_a_db, text width=3cm, below=of app] (llmserver) {Local\\LLM Server\\ Open Source};

\node[cloud_a, aspect=1.7, text width=2.5cm, left=3cm of llmserver] (model) {Open-Source\\LLM Models};

% Arrows
\draw[arrow] (browser) -- node[anchor=south] {HTTPS} (vscode);

\draw[arrow] (vscode) -- node[anchor=south] {Runs Code} (app);

\draw[arrow] (app) -- node[anchor=east] {Local API Calls} (llmserver);

\draw[arrow] (llmserver) -- node[anchor=south] {Loads} (model);

\end{tikzpicture}
}
  \vspace{0.5cm}
  \noindent

\begin{tikzpicture}[node distance=1.5cm]

    % Nodes with explicit positioning to prevent overlap
    \node[rectangle_a_blue, align=left, text width=3cm] (ip) {RAG chunks\\User Input-\\Topic\\Difficulty\\Blooms level};
    \node[rec_b_pink, right=3cm of ui] (llm) {LLM};
    \node[cloud_a, text width=1.8cm, right=2cm of llm] (op) {Initial MCQs};
    
    % Arrows
    \draw[arrow] (ip) -- node[anchor=south] {Prompt} (llm);
    \draw[arrow] (llm) -- (op);
    
\end{tikzpicture}}
  \vspace{0.5cm}
  \noindent

\begin{tikzpicture}[node distance=1.5cm]

    % Nodes with explicit positioning to prevent overlap
    \node[rectangle_a_blue, align=left, text width=6cm] (ip) {Initial MCQ drafts\\CoT reasoning instructions};
    \node[rec_b_pink, right=3cm of ui] (llm) {LLM};
    \node[cloud_a, text width=1.8cm, right=2cm of llm] (op) {Refined MCQs};
    
    % Arrows
    \draw[arrow] (ip) -- node[anchor=south] {Prompt} (llm);
    \draw[arrow] (llm) -- (op);
    
\end{tikzpicture}}

  \clearpage
  \begingroup

    % ---------- Headers ----------
    \def\cellOneA{\textbf{Tool}}
    \def\cellOneB{\textbf{Input}}
    \def\cellOneC{\textbf{Output}}
    \def\cellOneD{\textbf{Description}}
    
    \begin{longtable}{|>{\raggedright\arraybackslash}p{2.5cm}|>{\raggedright\arraybackslash}p{3.5cm}|>{\raggedright\arraybackslash}p{4.5cm}|>{\raggedright\arraybackslash}p{6cm}|} \hline
        \cellOneA & \cellOneB & \cellOneC & \cellOneD \\ \hline
        
        \textbf{CVM Prompt Templates} & 
        Submitted Python code & 
        CVM prompt for GPT-4o & 
        LangChain fills template to check valid Python \textcolor{green}{[EXPLICIT]} \\ \hline
        
        \textbf{GPT-4o} & 
        CVM prompt & 
        Valid/invalid Python flag & 
        Validates code syntax \textcolor{green}{[EXPLICIT]} \\ \hline
        
        \textbf{Server-side Tests} & 
        Valid code + test cases & 
        Pass/fail results + syntax errors & 
        Runs test cases + compilation check \textcolor{green}{[EXPLICIT]} \\ \hline
        
        \textbf{SCC Prompt Templates} & 
        Code that passed tests + 4 quality criteria & 
        SCC prompt for GPT-4o & 
        LangChain adds quality check instructions \textcolor{green}{[EXPLICIT]} \\ \hline
        
        \textbf{GPT-4o (SCC)} & 
        SCC prompt & 
        Quality issues or "No issues found" & 
        Checks unnecessary code, hardcoding, etc. \textcolor{green}{[EXPLICIT]} \\ \hline
        
        \textbf{RNC Prompt Templates} & 
        Problem + code + solution & 
        RNC prompt for GPT-4o & 
        LangChain + Algorithm 1 creates review necessity prompt \textcolor{green}{[EXPLICIT]} \\ \hline
        
        \textbf{GPT-4o (RNC)} & 
        RNC prompt & 
        reviewLabel (yes/nocorrect/nomeaningless) & 
        Decides if review needed (temp=0.2) \textcolor{green}{[EXPLICIT]} \\ \hline
        
        \textbf{System Logic} & 
        reviewLabel & 
        Predefined message OR proceed & 
        Branches: congratulate/complete/continue \textcolor{green}{[EXPLICIT]} \\ \hline
        
        \textbf{Review Prompt Templates} & 
        Problem + code + solution (if yes) & 
        Review Generation prompt & 
        LangChain fills review comment template \textcolor{green}{[EXPLICIT]} \\ \hline
        
        \textbf{GPT-4o (Review)} & 
        Review Generation prompt & 
        Markdown review comments + line highlights & 
        Generates learner-friendly "Code to fix" feedback \textcolor{green}{[EXPLICIT]} \\ \hline
        
        \textbf{Next.js UI} & 
        All results + comments & 
        Final student interface & 
        Renders Monaco Editor with annotations \textcolor{green}{[EXPLICIT]} \\ \hline
        
    \end{longtable}
\endgroup
}

  \clearpage
  \input{\subfix{sections/section_02_papers_innovation.tex}}

  % \clearpage
  % \section{MCQGen Paper Innovations Mapped to Pipeline Stages}
        \begin{enumerate}
            \item \textbf{Context-aware prompt templates integrating RAG retrieval with user specifications}
            \begin{itemize}
                \item \textit{Stage:} Stage-9: Combined Prompt Construction
                \item \textit{Explicit:} Authors designed specialized prompt templates that combine retrieved knowledge chunks with user-configured parameters (topic, difficulty, Bloom's taxonomy) \textbf{[EXPLICIT]}
                \item \textit{Inferred:} Specific merging logic for creating the combined prompt structure \textcolor{red}{\textbf{[INFERRED]}}
            \end{itemize}

            \item \textbf{RAG-enhanced LLM MCQ generation (core system novelty)}
            \begin{itemize}
                \item \textit{Stage:} Stage-10: MCQ Generation (Initial draft and Context-Aware)
                \item \textit{Explicit:} First system to use retrieval-augmented generation specifically for personalized MCQ creation from educational content \textbf{[EXPLICIT]}
                \item \textit{Inferred:} Operational API call structure using combined prompt \textcolor{red}{\textbf{[INFERRED]}}
            \end{itemize}

            \item \textbf{Chain-of-Thought (CoT) prompt engineering tailored for MCQ generation}
            \begin{itemize}
                \item \textit{Stage:} Stage-11: Prompt Construction for Chain-of-Thought
                \item \textit{Explicit:} Explicit use of CoT prompting to improve reasoning quality in question stem creation, correct answer identification, and distractor plausibility \textbf{[EXPLICIT]}
                \item \textit{Inferred:} Separate prompt construction step before CoT LLM call \textcolor{red}{\textbf{[INFERRED]}}
            \end{itemize}

            \item \textbf{Two-stage LLM generation (initial $\rightarrow$ CoT-refined MCQs)}
            \begin{itemize}
                \item \textit{Stage:} Stage-12: MCQ Generation (refined using Chain-of-Thought Guided)
                \item \textit{Explicit:} Iterative refinement using CoT prompting produces measurably better MCQs than single-pass generation \textbf{[EXPLICIT]}
                \item \textit{Inferred:} Specific implementation as separate API call with CoT prompt \textcolor{red}{\textbf{[INFERRED]}}
            \end{itemize}

            \item \textbf{Iterative self-critique and correction loop for MCQ quality (NEW STAGE)}
            \begin{itemize}
                \item \textit{Stage:} Stage-13: Self-Refine MCQ Refinement \textbf{(NEW STAGE)}
                \item \textit{Explicit:} LLM-based self-refinement using expert-designed scoring rubrics, cycling through critique $\rightarrow$ correction $\rightarrow$ re-evaluation \textbf{[EXPLICIT]}
                \item \textit{Explicit:} Multiple iterations per MCQ until quality thresholds met \textbf{[EXPLICIT]}
                \item \textit{Inferred:} Per-MCQ processing loop structure \textcolor{red}{\textbf{[INFERRED]}}
            \end{itemize}

            \item \textbf{Multi-dimensional expert-aligned MCQ evaluation rubric (NEW COMPLETE STAGE)}
            \begin{itemize}
                \item \textit{Stage:} Stage-14: MCQ Validation and Evaluation \textbf{(NEW COMPLETE STAGE)}
                \item \textit{Explicit:} 5-criteria scoring (clarity, relevance, difficulty, answer correctness, distractor quality) validated against human educators \textbf{[EXPLICIT]}
                \item \textit{Explicit:} Automated metrics (BLEU, ROUGE) + human expert validation \textbf{[EXPLICIT]}
            \end{itemize}

            \item \textbf{User-configurable MCQ parameters mapped to Bloom's taxonomy}
            \begin{itemize}
                \item \textit{Stage:} Cross-stage personalization (primarily Stage-9)
                \item \textit{Explicit:} Dynamic adjustment of difficulty, cognitive level, and question count based on user input \textbf{[EXPLICIT]}
                \item \textit{Inferred:} Integration into Stage-9 prompt construction \textcolor{red}{\textbf{[INFERRED]}}
            \end{itemize}
        \end{enumerate}

\textbf{Summary:} MCQGen introduces \textbf{5 modified stages + 2 completely new stages}, concentrating innovations in the MCQ Generation and Refinement block while building on standard RAG preprocessing (Stages 1-8 remain conventional) [web:14].}

  \clearpage
  \section{Novel Research Directions (2025+ Publication Potential)}

\begin{enumerate}
    \item Build similar system for javascript, html, css.
\end{enumerate}


\section{Papers}
	\begin{enumerate}
		\item Facilitating university admission using a chatbot based on large language models with retrieval-augmented generation. \url{https://doi.org/10.30191/ETS.202410_27(4).TP02}
		\item a QA chatbot for rgpv diploma students . SYLLABUSQA: A Course Logistics Question Answering Dataset. 
		\item LLM html. css, js, python, c java programming.  Enhancing Python Learning Through Retrieval-Augmented Generation: A Theoretical and Applied Innovation in Generative AI Education. 
		\item HICON AI: Higher Education Counseling Bot
		\item Yaying Huang Automatic lesson plan generation via large language models with self-critique prompting
	\end{enumerate}}

  \clearpage
  \section{Code Review Pipeline - LangChain Prompt-LLM Pattern}

The Code Review system follows a systematic \textbf{prompt $\rightarrow$ LLM $\rightarrow$ server $\rightarrow$ prompt $\rightarrow$ LLM} pattern using LangChain:

  \begin{enumerate}
      \item \textbf{CVM prompt $\rightarrow$ GPT-4o} $\rightarrow$ Valid Python code
      \item \textbf{Server-side tests} $\rightarrow$ Pass/fail results
      \item \textbf{SCC prompt $\rightarrow$ GPT-4o} $\rightarrow$ Quality issues
      \item \textbf{LangChain RNC PROMPT TEMPLATES} $\rightarrow$ RNC prompt  
      \item \textbf{GPT-4o} $\rightarrow$ reviewLabel (yes/nocorrect/nomeaningless)
      \item \textbf{Decision Branch} $\rightarrow$ (if yes)
      \item \textbf{LangChain REVIEW PROMPT TEMPLATES} $\rightarrow$ Review Generation prompt
      \item \textbf{GPT-4o} $\rightarrow$ Learner-friendly review comments
      \item Next.js $\rightarrow$ Final UI output
  \end{enumerate}

\textcolor{green}{[EXPLICIT]} LangChain-orchestrated LLM calls with server-side filtering. [file:1]
}
  
  \clearpage
  \section{Stage-1: Regex-Based Character Filtering and Normalization}
    \begin{enumerate}
        \item \textbf{Description:} The MCQGen paper does NOT provide explicit details about regex-based character filtering or text normalization. The paper acknowledges that a ``knowledge base'' is constructed from external sources but abstracts away low-level text cleaning techniques. 
        
        \item \textbf{Input:} Raw lesson text.
        
        \item \textbf{Output:} \textcolor{red}{\textbf{[INFERRED]}} Formatted and cleaned text ready for the next stage, with:
          \begin{enumerate}
            \item All HTML tags removed, any text between \verb|<| and \verb|>|.
              \item Special characters filtered out (e.g., \texttt{\$, @, \#, \%, \&, *}).
              \item Whitespace normalized (single spaces, leading and trailing whitespace removed, consistent line breaks).
              \item Fixing spaces around punctuation(. , ? !  : ; etc).
          \end{enumerate}
        
        \item \textbf{Tool/Method/Algorithm:} \textcolor{red}{\textbf{[INFERRED]}} Regex (Regular Expression) pattern matching and substitution.

        \item \textbf{Process:} \textcolor{red}{\textbf{[INFERRED]}} Apply a sequence of regex patterns in order to clean the input text.
    \end{enumerate}}
  \input{\subfix{sections/section_05_pipeline_stage_02.tex}}
  
  \clearpage
  \section{Stage-3: Text Tokenization}

\begin{enumerate}
    \item \textbf{Description:} The MCQGen paper does not explicitly describe a separate tokenization step, but any NLP pipeline that converts text into embeddings or uses LLMs assumes that the text is tokenized before further processing.[web:107][web:151] \textcolor{red}{\textbf{[INFERRED]}}

    \item \textbf{Input:} Formatted lesson text from previous stage.
    
    \item \textbf{Output:}  A ordered sequence(list) of tokens representing the text, where each token is typically a word or subword unit, ready for chunking/segmentation or direct embedding. \textcolor{red}{\textbf{[INFERRED]}}
    
    \item \textbf{Tool/Method/Algorithm:} \textcolor{red}{\textbf{[INFERRED]}} A tokenizer (e.g., whitespace/punctuation-based tokenizer, or the LLM’s own subword tokenizer such as BPE or WordPiece).[web:154][web:159]
    
    \item \textbf{Process:} \textcolor{red}{\textbf{[INFERRED]}} Split the input text into tokens using a tokenizer
        \begin{itemize}
            \item For simple word tokenization, split on spaces and punctuation so that each meaningful word becomes a token.[web:154][web:161]
            \item For model-specific tokenization, use the LLM’s tokenizer to convert the text into subword tokens that match the model’s vocabulary.[web:159]
        \end{itemize}
\end{enumerate}
}
  \section*{Stage-4: Text Chunking/Segmentation}

\begin{enumerate}
    \item \textbf{Description:} The MCQGen paper does not explicitly detail chunking or segmentation; however, chunking is a standard preprocessing step in NLP especially for retrieval-augmented systems to split text into manageable pieces for embedding and retrieval purposes. \textcolor{red}{\textbf{[INFERRED]}}

    \item \textbf{Input:} Tokenized lesson text from previous stage. \textcolor{red}{\textbf{[INFERRED]}}
    
    \item \textbf{Output:} \textcolor{red}{\textbf{[INFERRED]}} List (or collection) of chunks. Each chunk
        \begin{enumerate}
            \item Represents a semantically coherent segment of the text
            \item can represent a sentence or paragraph(logically grouped set of sentences).
            \item Is designed to capture a meaningful piece of information
        \end{enumerate}
    
    \item \textbf{Tool/Method/Algorithm:}  \textcolor{red}{\textbf{[INFERRED]}} Chunking algorithms or heuristics based on sentence boundaries, fixed token count windows, sliding windows, or semantic coherence measures.
    
    \item \textbf{Process:} Segmentation is done using one or a combination of the following:
        \begin{itemize}
            \item Sentence boundary detection to keep chunks aligned with natural sentences.
            \item Fixed-size token windows with overlapping sliding windows to capture context.
            \item Semantic or topical coherence measures to group related tokens.
        \end{itemize}

\end{enumerate}
}
  
  \clearpage
  \section{Stage-5: Embedding/Vectorization}

\begin{enumerate}
    \item \textbf{Description:} \textcolor{green}{\textbf{[EXPLICIT]}} The MCQGen paper explicitly mentions the use of vector embeddings as part of its Retrieval-Augmented Generation (RAG) framework to represent text chunks for semantic similarity search and retrieval.
    
    \item \textbf{Input:}  A list of semantically coherent text chunks. \textcolor{red}{\textbf{[INFERRED]}} The paper does not detail chunking, but this input is a logical prerequisite for embedding.
    
    \item \textbf{Output:} \textcolor{green}{\textbf{[EXPLICIT]}} A list of dense vector embeddings representing each text chunk in a semantic vector space, suitable for efficient similarity computations and retrieval. 
    
    \item \textbf{Tool/Method/Algorithm:}  \textcolor{green}{\textbf{[EXPLICIT]}} Pretrained embedding models or encoders (e.g., BERT, Sentence-BERT, or proprietary embedding models) are used to convert text chunks into vector representations.
    
    \item \textbf{Process:}
    \begin{itemize}
        \item \textcolor{green}{\textbf{[EXPLICIT]}} Each text chunk is passed through a pretrained embedding model that transforms the natural language text into a fixed-length numeric vector capturing semantic meaning.
        \item \textcolor{red}{\textbf{[INFERRED]}} Embeddings are normalized/scaled as needed for downstream vector similarity computations.
        \item \textcolor{green}{\textbf{[EXPLICIT]}} The resulting vectors form the basis of the Vector Database used in Stage-6 for semantic retrieval.
    \end{itemize}
\end{enumerate}
}
  
  \clearpage
  \input{\subfix{sections/section_09_pipeline_stage_06.tex}}
  
  \clearpage
  \input{\subfix{sections/section_10_pipeline_stage_07.tex}}
  
  \clearpage
  \section{Stage-8: Semantic Retrieval of chunks from vector DB}
      \begin{enumerate}
          \item \textbf{Description:} \textcolor{green}{\textbf{[EXPLICIT]}} The MCQGen paper explicitly discusses the use of Retrieval-Augmented Generation (RAG), which involves retrieving relevant knowledge from a vector database based on semantic similarity between the query and stored embeddings. 
          
          \item \textbf{Input:} \textcolor{red}{\textbf{[INFERRED]}} 
          \begin{itemize}
              \item Semantic query embedding. 
              \item Vector-indexed knowledge base.
          \end{itemize}
          
          \item \textbf{Output:} \textcolor{green}{\textbf{[EXPLICIT]}} A ranked list of retrieved text chunks or documents relevant to the query, which provides contextual knowledge for MCQ generation. 
          
          \item \textbf{Tool/Method/Algorithm:} \textcolor{green}{\textbf{[EXPLICIT]}} Vector similarity search algorithms (e.g., cosine similarity)
          
          \item \textbf{Process:}
          \begin{itemize}
              \item \textcolor{green}{\textbf{[EXPLICIT]}} Perform a nearest neighbor search in the vector database using the semantic query embedding.
              \item \textcolor{red}{\textbf{[INFERRED]}} Rank the retrieved results based on similarity scores.
              \item \textcolor{green}{\textbf{[EXPLICIT]}} Return the top relevant chunks as context for MCQ generation in Stage-9.
          \end{itemize}
      \end{enumerate}}

  \clearpage
  \section{Stage-9: Construction of Custom Combined Prompt}
      \begin{enumerate}
          \item \textbf{Description:} This stage constructs a enhanced  prompt that is ready for submission as input to the LLM API. The purpose of prompt templates is to shape and constrain the LLM’s generation behavior, ensuring the production of well-structured and meaningful MCQs. \textcolor{green}{\textbf{[EXPLICIT]}}
          
          \item \textbf{Input:} 
          \begin{itemize}
              \item User’s original instruction \textcolor{green}{\textbf{[EXPLICIT]}}
              \item Context( retrieved chunks) \textcolor{red}{\textbf{[INFERRED]}}
              \item Predefined prompt templates or engineered instructions specifying output format and content. \textcolor{green}{\textbf{[EXPLICIT]}}
          \end{itemize}
          
          \item \textbf{Output:} \textcolor{red}{\textbf{[INFERRED]}} Custom Combined Prompt, ready for LLM API submission.
          
          \item \textbf{Tool/Method/Algorithm:}
          \begin{itemize}
              \item Construction logic, merging text blocks via string concatenation, templating engines, or prompt assembly frameworks. \textcolor{red}{\textbf{[INFERRED]}}
          \end{itemize}
          
          \item \textbf{Process:}
          \begin{itemize}
              \item \textcolor{red}{\textbf{[INFERRED]}} Combine the user’s configuration (topic, difficulty, number of questions) with the text chunks retrieved from the vector database.  
              \item \textcolor{green}{\textbf{[EXPLICIT]}} Integrate prompt templates that specify how questions, answers, and distractors should be formatted and generated.  
              \item \textcolor{red}{\textbf{[INFERRED]}} Format the combined content into a coherent, structured prompt suitable for the LLM input.  
              \item \textcolor{red}{\textbf{[INFERRED]}} Output the final prompt string to be used for LLM querying.  
          \end{itemize}
      \end{enumerate}}
  \input{\subfix{sections/section_13_pipeline_stage_10.tex}}

  \clearpage
  \input{\subfix{sections/section_14_pipeline_stage_11.tex}}
  \input{\subfix{sections/section_15_pipeline_stage_12.tex}}

  \clearpage
  \section{Stage-13: Prompt Construction for LLM to Generate Scores and Instructions}
    \begin{enumerate}
        \item \textbf{Description:} This stage creates a special prompt that asks the LLM to critique its own generated stem and key for difficulty level without changing the correct answer. \textcolor{green}{\textbf{[EXPLICIT]}}

        \item \textbf{Input:} 
            \begin{itemize}
                \item The current/refined MCQs (stem + key). \textcolor{red}{\textbf{[INFERRED]}}
                \item Self-refine critique prompt templates. \textcolor{green}{\textbf{[EXPLICIT]}}
                \item Special instructions for LLM to score difficulty (1-5) and suggest enhancements. \textcolor{green}{\textbf{[EXPLICIT]}}
            \end{itemize}

        
        \item \textbf{Output:} \textcolor{red}{\textbf{[INFERRED]}} A critique prompt to be sent to LLM in next stage.
        
        \item \textbf{Tool/Method/Algorithm:} Crafting text prompts to include self-critique instructions with current MCQs. \textcolor{red}{\textbf{[INFERRED]}}
        
        \item \textbf{Process:} 
        \begin{itemize}
            \item Combine the refined MCQs with self-refine critique prompt templates from Table 2. \textcolor{green}{\textbf{[EXPLICIT]}}
            \item Format the combined prompt to guide LLM to generate scores + improvement suggestions. \textcolor{red}{\textbf{[INFERRED]}}
        \end{itemize}
    \end{enumerate}}
  \section{Stage-14: LLM Generates Scores and Instructions (Critique)}
    \begin{enumerate}
        \item \textbf{Description:} This stage sends the self-refine critique prompt to the LLM via API call to generate difficulty scores and improvement suggestions. \textcolor{green}{\textbf{[EXPLICIT]}}

        \item \textbf{Input:} The self-refine critique prompt from Stage-13. \textcolor{red}{\textbf{[INFERRED]}}
        
        \item \textbf{Output:} \textcolor{green}{\textbf{[EXPLICIT]}} For each MCQ, Scores (1-5 difficulty) and critique instructions/suggestions.
        
        \item \textbf{Tool/Method/Algorithm:} LLM (GPT-4). \textcolor{red}{\textbf{[INFERRED]}}
        
        \item \textbf{Process:} Perform API call to LLM and send self-refine critique prompt. \textcolor{red}{\textbf{[INFERRED]}}
    \end{enumerate}}

  \clearpage
  \section{Stage-15: Prompt Construction for LLM to Refine the MCQs}
    \begin{enumerate}
        \item \textbf{Description:} This stage creates a special prompt that asks the LLM to integrate critique scores and suggestions to revise the stem while keeping the correct answer unchanged. \textcolor{green}{\textbf{[EXPLICIT]}}

        \item \textbf{Input:} 
        \begin{itemize}
            \item The current/refined MCQs (stem + key). \textcolor{red}{\textbf{[INFERRED]}}
            \item Scores (1-5 difficulty) and critique instructions/suggestions from Stage-14. \textcolor{green}{\textbf{[EXPLICIT]}}
            \item Self-refine refinement prompt templates. \textcolor{green}{\textbf{[EXPLICIT]}}
        \end{itemize}
        
        \item \textbf{Output:} \textcolor{red}{\textbf{[INFERRED]}} A refinement prompt to be sent to LLM in next stage.
        
        \item \textbf{Tool/Method/Algorithm:} Crafting text prompts to combine MCQs with critique feedback. \textcolor{red}{\textbf{[INFERRED]}}
        
        \item \textbf{Process:} 
        \begin{itemize}
            \item Combine current MCQs with scores, critique instructions, and self-refine refinement templates from Table 2. \textcolor{green}{\textbf{[EXPLICIT]}}
            \item Format the combined prompt to guide LLM to revise stem for higher difficulty. \textcolor{red}{\textbf{[INFERRED]}}
        \end{itemize}
    \end{enumerate}}
  \section{Stage-16: LLM Generates Refined MCQs}
    \begin{enumerate}
        \item \textbf{Description:} This stage sends the self-refine MCQ refinement prompt to the LLM via an API call, prompting it to revise the stem for higher difficulty while keeping the correct answer unchanged. Loop continues until LLM scores reach target difficulty OR instructor approves final stem. \textcolor{green}{\textbf{[EXPLICIT]}}

        \item \textbf{Input:} The self-refine refinement prompt from Stage-15. \textcolor{red}{\textbf{[INFERRED]}}
        
        \item \textbf{Output:} \textcolor{green}{\textbf{[EXPLICIT]}} Refined MCQs with updated stems and unchanged keys.
        
        \item \textbf{Tool/Method/Algorithm:} LLM (GPT-4). \textcolor{red}{\textbf{[INFERRED]}}
        
        \item \textbf{Process:} Perform API call to LLM and send self-refine MCQ refinement prompt. \textcolor{red}{\textbf{[INFERRED]}}
    \end{enumerate}}

  \clearpage
  \section{Stage-17: MCQ Validation and Evaluation}
        \begin{enumerate}
            \item \textbf{Description:} This stage evaluates the quality of refined MCQs using both automatic metrics and human expert review. The MCQGen paper explicitly describes validation through expert scoring on criteria like clarity, difficulty, relevance, and distractor plausibility, comparing LLM-generated MCQs against human-created benchmarks. \textbf{[EXPLICIT]}

            \item \textbf{Input:} 
            \begin{itemize}
                \item Final refined MCQs from Stage-13 self-refinement process. \textcolor{red}{\textbf{[INFERRED]}}
                \item Automated evaluation metrics (BLEU, ROUGE, perplexity) and expert scoring rubrics. \textbf{[EXPLICIT]}
            \end{itemize}
            
            \item \textbf{Output:} 
            \begin{itemize}
                \item Quality scores and validation reports for each MCQ, with pass/fail status for final inclusion. \textcolor{red}{\textbf{[INFERRED]}}
                \item Approved MCQs ready for formatting and delivery. \textbf{[EXPLICIT]}
            \end{itemize}
            
            \item \textbf{Tool/Method/Algorithm:} 
            \begin{itemize}
                \item Automated metrics: BLEU, ROUGE, semantic similarity measures. \textbf{[EXPLICIT]}
                \item Human expert evaluation using predefined rubrics aligned with educational standards. \textbf{[EXPLICIT]}
            \end{itemize}
            
            \item \textbf{Process:} 
            \begin{itemize}
                \item Run automated metrics (BLEU, ROUGE, perplexity) against reference standards. \textbf{[EXPLICIT]}
                \item Conduct expert human evaluation using scoring rubrics on key criteria (clarity, difficulty, relevance). \textbf{[EXPLICIT]}
                \item Compare LLM-generated MCQs against human-created benchmarks for quality alignment. \textbf{[EXPLICIT]}
                \item Flag MCQs failing thresholds for exclusion or additional refinement. \textcolor{red}{\textbf{[INFERRED]}}
                \item Output validated MCQs approved for final formatting and use. \textcolor{red}{\textbf{[INFERRED]}}
            \end{itemize}
        \end{enumerate}}
  \clearpage


  \section{Imp points}
    \begin{enumerate}
        \item Ask chatgpt what steps(Algorithm, stages) it uses to create a mcq question from lesson text.
        \item techinque of this paper is to create a prompt for the LLM to get a mcq. This prompt will contain a concept and its destractors.
        \item Improve mcqGen for multiple correct choices, currently it has only one correct answers.
    \end{enumerate} 


      \section{Ideas}
        \begin{enumerate}
          \item Technique to create distractors for mcq. 
          \item Use the technique presented in MCQ gen paper to create(generate) explanation/tutorial of a research paper. 
          \item Improve mcqGen for multiple correct choices, currently it has only one correct answers.
          \item Propose RAG for MCQ, fill in the blank or other types of questions.
          \item Propose chunking stratgy for  MCQ, fill in the blank or other types of questions.
          \item Propose  a preprocessing step for MCQ, fill in the blank or other types of questions.
          \item method to extract topics, keywords, important phrases etc from a given lesson text. Ask chatgpt what algorithm it uses to extract topics, keywords, important phrases etc from a given lesson text. Do some innovation in it.
          \item do a comparative study about 2 techniques in RAG process for generating MCQ.
          \item propose  a methods to extract topics/keywords/correct answers and distractors for mcq generation.
          \item  generate a list of questions to understand a topic.
          \item propose a new prompt creation technique to extract MCQs.
          \item give a chapter of CBSE subjects EVS, Moral science, english and generate questions and compare with CBSE books.
          \item Content generation in hindi given a english text.
          \item MCQ generation based on gaps identified for an essay written by student
          \item given an MCQ question using RAG to find alternate multiple choice options.
          \item generate concepts which forms the basis for multiple choice MCQs.
          \item generate keywords, topics using iterative self refining steps and CoT. 
        
        \end{enumerate}

    \section{Tools List}
    \begin{enumerate}
      \item PDF-to-text engine
      \item DOCX parser (paper mentions generic preprocessing, not specific software)
      \item Sentence segmentation algorithm : create a list of sentences from a big text.
      \item Sentence grouping logic : group sentences into chunks
      \item Semantic similarity check (lightweight) : Validate each chunk covers one idea or sub-topic Tokenizer
      \item Tokenizer
      \item Stopword filter
      \item Lemmatizer
      \item TF-IDF
      \item RAKE algorithm
      \item TextRank algorithm
      \item Keyword merging and scoring script
      \item NER module
      \item LDA (Latent Dirichlet Allocation) or similar method, Topic modeling engine
      \item Semantic similarity / relevance scoring
    \end{enumerate}

\end{document}