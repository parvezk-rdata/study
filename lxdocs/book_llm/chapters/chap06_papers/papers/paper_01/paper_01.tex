\documentclass[../../../../main.tex]{subfiles}
\begin{document}


% \input{\subfix{images/Embedding_Models.tex}}
% \begin{simpleTable}{Values and Their meaning}   
    
    \def\cellThreeB{\makecell[l]{Small document.\\Low latency, fast response}}
     
    \begin{tabular}{|c|c|c|} \hline
        LLM Models 		                & Nano, mini, flash		& GPT 4.1, Gemini 2.5 pro 				\\ \hline
        Context window size(tokens)		& 2k to 4k	            & 1 Million		\\ \hline
        Best for			            & \cellThreeB           & Large document				\\ \hline
    \end{tabular}
    
\end{simpleTable}}

\chapter{MCQGen: A Large Language Model-Driven MCQ Generator for Personalized Learning} 
	
	\section{Information }
	
	\begin{enumerate}
		\item \cite{Hang2024}
    \item IEEE Access , DOI : 10.1109/ACCESS.2024.3420709
		\item \url{https://drive.google.com/file/d/1KkfsKJhbx3S2hLeA6iMZsXpkQdwaRPMb/view?usp=drive_link}
	\end{enumerate}


  \section{Pipeline stages which converts input to final output. }
  \begin{enumerate}\item \textbf{Knowledge Base Preparation}
    \begin{enumerate}
        \item Stage-1: Text Preprocessing and Cleaning
        
        \item Stage-2: Text Tokenization
        
        \item Stage-3: Text Chunking/Segmentation
        
        \item Stage-4: Embedding/Vectorization
        
        \item Stage-5: Vector Database Indexing
    \end{enumerate}
    
    \item \textbf{Query Processing and Retrieval}
    \begin{enumerate}
        \item Stage-6: Query Formulation
        
        \item Stage-7: Semantic Retrieval
    \end{enumerate}
    
    \item \textbf{MCQ Generation and Refinement}
    \begin{enumerate}
        \item Stage-8: Context-Aware MCQ Generation using LLM
        
        \item Stage-9: Chain-of-Thought Prompt Engineering
        
        \item Stage-10: Self-Refine MCQ Refinement
    \end{enumerate}
    
    \item \textbf{Quality Assurance and Output}
    \begin{enumerate}
        \item Stage-11: MCQ Validation and Evaluation
        
        \item Stage-12: Final Output Formatting
    \end{enumerate}
  \end{enumerate}

  \section*{Stage-1: Regex-Based Character Filtering and Normalization}
    \begin{enumerate}
        \item \textbf{Description:} The MCQGen paper does NOT provide explicit details about regex-based character filtering or text normalization. The paper acknowledges that a ``knowledge base'' is constructed from external sources but abstracts away low-level text cleaning techniques. 
        
        \item \textbf{Input:} Raw lesson text.
        
        \item \textbf{Output:} \textcolor{red}{\textbf{[INFERRED]}} Formatted and cleaned text ready for the next stage, with:
          \begin{enumerate}
              \item All HTML tags removed, e.g., using pattern \texttt{<[^>]*>} to match any text between \texttt{<} and \texttt{>}.
              \item Special characters filtered out (e.g., \texttt{\$, @, \#, \%, \&, *}) using a pattern such as \texttt{[\$@\#\%\&*]}.
              \item Whitespace normalized (single spaces, leading and trailing whitespace removed, consistent line breaks).
              \item Punctuation properly spaced.
              \item Text ready for encoding standardization in Stage-2.
          \end{enumerate}
        
        \item \textbf{Tool/Method/Algorithm:} \textcolor{red}{\textbf{[INFERRED]}} Regex (Regular Expression) pattern matching and substitution.

        \item \textbf{Process:} \textcolor{red}{\textbf{[INFERRED]}} Apply a sequence of regex patterns in order to clean the input text.
    \end{enumerate}


  \section{Imp points}
    \begin{enumerate}
        \item Ask chatgpt what steps(Algorithm, stages) it uses to create a mcq question from lesson text.
        \item techinque of this paper is to create a prompt for the LLM to get a mcq. This prompt will contain a concept and its destractors.
        \item Improve mcqGen for multiple correct choices, currently it has only one correct answers.
    \end{enumerate} 


      \section{Ideas}
        \begin{enumerate}
          \item Technique to create distractors for mcq. 
          \item Use the technique presented in MCQ gen paper to create(generate) explanation/tutorial of a research paper. 
          \item Improve mcqGen for multiple correct choices, currently it has only one correct answers.
        \end{enumerate}

    \section{Tools List}
    \begin{enumerate}
      \item PDF-to-text engine
      \item DOCX parser (paper mentions generic preprocessing, not specific software)
      \item Sentence segmentation algorithm : create a list of sentences from a big text.
      \item Sentence grouping logic : group sentences into chunks
      \item Semantic similarity check (lightweight) : Validate each chunk covers one idea or sub-topic Tokenizer
      \item Tokenizer
      \item Stopword filter
      \item Lemmatizer
      \item TF-IDF
      \item RAKE algorithm
      \item TextRank algorithm
      \item Keyword merging and scoring script
      \item NER module
      \item LDA (Latent Dirichlet Allocation) or similar method, Topic modeling engine
      \item Semantic similarity / relevance scoring
    \end{enumerate}

\end{document}