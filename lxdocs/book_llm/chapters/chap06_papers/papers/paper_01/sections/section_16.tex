\section{MCQGen Paper Innovations Mapped to Pipeline Stages}
        \begin{enumerate}

            \item \textbf{Uses of RAG to enhanced LLM for MCQ generation (core system novelty)}
            \begin{itemize}
                \item \textit{Stage:} Stage-10: MCQ Generation (Initial draft and Context-Aware)
                \item \textit{Explicit:} First system to use RAG specifically for MCQ creation. Used semantically retrieved chunks rather than model knowledge alone \textbf{[EXPLICIT]}
                \item \textit{Inferred:} Operational API call structure using combined prompt \textcolor{red}{\textbf{[INFERRED]}}
            \end{itemize}

            \item \textbf{Use of Context-aware prompt templates}
            \begin{itemize}
                \item \textit{Stage:} Stage-9: Combined Prompt Construction
                \item \textit{Explicit:} Authors designed specialized prompt templates that combine retrieved knowledge chunks with user-configured parameters (topic, difficulty, Bloom's taxonomy) \textbf{[EXPLICIT]}
                \item \textit{Inferred:} Specific merging logic for creating the combined prompt structure \textcolor{red}{\textbf{[INFERRED]}}
            \end{itemize}

            \item \textbf{Chain-of-Thought (CoT) prompt engineering tailored for MCQ generation}
            \begin{itemize}
                \item \textit{Stage:} Stage-11: Prompt Construction for Chain-of-Thought
                \item \textit{Explicit:} Explicit use of CoT prompting to improve reasoning quality in MCQ creation.  Enable LLM to "think step-by-step" \textbf{[EXPLICIT]}
                \item \textit{Inferred:} Separate prompt construction step before CoT LLM call \textcolor{red}{\textbf{[INFERRED]}}
            \end{itemize}

            \item \textbf{Two-stage LLM generation (initial $\rightarrow$ CoT-refined MCQs)}
            \begin{itemize}
                \item \textit{Stage:} Stage 10 and 12: MCQ Generation (refined using Chain-of-Thought Guided)
                \item \textit{Explicit:} Created MCQ in 2 passes. First pass generated initial draft and second pass refined them using CoT. \textbf{[EXPLICIT]}
                \item \textit{Inferred:} Specific implementation as separate API call with CoT prompt \textcolor{red}{\textbf{[INFERRED]}}
            \end{itemize}

            \item \textbf{Iterative self-critique and correction loop for MCQ quality (NEW STAGE)}
            \begin{itemize}
                \item \textit{Stage:} Stage-13: Self-Refine MCQ Refinement \textbf{(NEW STAGE)}
                \item \textit{Explicit:}  LLM cycled through critique $\rightarrow$ correction $\rightarrow$ re-evaluation. Used expert-designed scoring rubrics for self-refinement. \textbf{[EXPLICIT]}
                \item \textit{Explicit:} Multiple iterations per MCQ until quality thresholds met \textbf{[EXPLICIT]}
                \item \textit{Inferred:} Per-MCQ processing loop structure \textcolor{red}{\textbf{[INFERRED]}}
            \end{itemize}

            \item \textbf{Use of Rubric for MCQ validation and evaluation. Multi-dimensional expert-aligned rubric (NEW COMPLETE STAGE)}
            \begin{itemize}
                \item \textit{Stage:} Stage-14: MCQ Validation and Evaluation \textbf{(NEW COMPLETE STAGE)}
                \item \textit{Explicit:} 5-criteria scoring (clarity, relevance, difficulty, answer correctness, distractor quality) validated against human educators \textbf{[EXPLICIT]}
                \item \textit{Explicit:} Automated metrics (BLEU, ROUGE) + human expert validation \textbf{[EXPLICIT]}
            \end{itemize}

            \item \textbf{User-configurable MCQ parameters mapped to Bloom's taxonomy}
            \begin{itemize}
                \item \textit{Stage:} Cross-stage personalization (primarily Stage-9)
                \item \textit{Explicit:} Dynamic adjustment of difficulty, cognitive level, and question count based on user input \textbf{[EXPLICIT]}
                \item \textit{Inferred:} Integration into Stage-9 prompt construction \textcolor{red}{\textbf{[INFERRED]}}
            \end{itemize}
        \end{enumerate}

\textbf{Summary:} MCQGen introduces \textbf{5 modified stages + 2 completely new stages}, concentrating innovations in the MCQ Generation and Refinement block while building on standard RAG preprocessing (Stages 1-8 remain conventional) [web:14].