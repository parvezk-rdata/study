\section{Stage-7: Query Formulation}
    \begin{enumerate}
        \item \textbf{Description:} \textcolor{red}{\textbf{[INFERRED]}} The MCQGen paper does not explicitly describe the query formulation step. However, it is a necessary part of the Retrieval-Augmented Generation framework, where user inputs such as topic, difficulty, and learning objectives are converted into a semantic search query to retrieve relevant text chunks.

        \item \textbf{Input:} \textcolor{red}{\textbf{[INFERRED]}} 
        \begin{itemize}
            \item User and instructor preferences (e.g., topic, difficulty level, Bloom’s taxonomy category, number of MCQs).
            \item Related metadata or configuration specifying the targeted knowledge scope.
        \end{itemize}
        
        \item \textbf{Output:} \textcolor{red}{\textbf{[INFERRED]}}  A semantic search query or query embedding crafted to represent user preferences for retrieving relevant knowledge chunks from the vector index.
        
        \item \textbf{Tool/Method/Algorithm:} \textcolor{red}{\textbf{[INFERRED]}} Natural language processing techniques for query construction (could involve prompt templates or embedding generation).
        
        \item \textbf{Process:} \textcolor{red}{\textbf{[INFERRED]}} 
        \begin{itemize}
            \item Extract relevant aspects of user input and educational goals.
            \item Convert input into a structured or unstructured semantic query.
            \item Generate the vector embedding of the query for similarity search against the vector database.
        \end{itemize}
    \end{enumerate}