\documentclass[../../main.tex]{subfiles}
\begin{document}

\chapter{Research prompts}

I am a beginner trying to read research papers with the aim of identifying research gaps so I can publish my own papers.\par However, I have no prior experience with research or reading academic papers, and I find it quite challenging due to several issues.\newline These include difficult English, complex paper structures, poorly explained content, and limited background knowledge on the topics discussed.\par Currently, I have decided to focus my research on AI/ML, specifically the use of large language models (LLMs) in education.\par I would like to discuss how you can assist me in understanding research papers.\newline It would be helpful if you could provide me with a list of clear, well-structured prompts that I can use to gradually build my comprehension---from the basics to a thorough understanding of any research paper.\par 
\clearpage

\section{My prompts}
	
  \begin{enumerate}
    \item TL;DR of this paper aimed at a novice school student(beginner-friendly) in point vise manner
    \item Summarize this paper in aimed at a beginner school student. Cover these aspects: the problem the paper addresses, why this problem matters, what the authors did to solve it, how they tested or evaluated their work, and the main result or finding.
    \item List all tools / technologies / methods / methodologies used in this paper (e.g., models,libraries, datasets, metrics), aimed at a beginner school student
    \item What is the the initial input and final output of the work propesed by this paper. aimed at a beginner school student.
    \item What the pipeline stages which converts input to final output. Give a step-by-step narrative explanation of each stage (like how data flows and decisions happen)? your answer should be aimed at a beginner school student
    \item Make a table with 5 columns: stage name, input to this stage, description, tools used, outputs of this stage. your answer should be aimed at a beginner school stu- dent.write latex code. the table should not be truncated in page. all collumns should be visibel.
  \end{enumerate}
  \clearpage
\section{Prompts}
	
	\begin{enumerate}
		\item Act as my research mentor. For each part or section, explain key ideas and ask me reflective questions so I can build understanding.
		\item Give a list of tools/technologies/methods/methodologies they used in this paper
		\item What is the the initial input and final output of the work propesed by this paper.
		\item What the stages which converts input(example a prompt) to final output(example mcq questions)
		\item Create a diagram/table/flowchart showing how all these components interact in the MCQGen pipeline?
		\item Convert this into a visual diagram (flowchart-style image) showing arrows and boxes
		\item Give a step-by-step narrative explanation of each stage (like how data flows and decisions happen)?
		\item What new thing the paper as done or proposed and what was the earlier approach
		\item Summarize this paper simply for a beginner
		\item Explain what each section of this paper contributes
		\item Explain the technical terms and equations simply.
		\item Explain how their experiment works and what they found
		\item Evaluate the paper’s methodology and reliability
		\item List research gaps or open problems from this paper
		\item Suggest related works or follow-up papers
		\item Based on these gaps, suggest possible research ideas for me
	\end{enumerate}

	\clearpage

	\section*{Prompt Categories for Understanding Research Papers}

\begin{enumerate}[label=\textbf{\Alph*.}, leftmargin=2em]

\item \textbf{Quick / Beginner Prompts}
  \begin{itemize}
    \item TL;DR of this paper aimed at a novice school student(beginner-friendly) in point vise manner.
	\item Give a 2-sentence explanation aimed at a high-school student
    \item Summarize this paper in aimed at a beginner school student. Cover these aspects: the problem the paper addresses, why this problem matters, what the authors did to solve it, how they tested or evaluated their work, and the main result or finding.
    \item Translate the abstract into plain English. Replace technical words with everyday words.
    \item List all glossary terms from this paper and explain each, aimed at a beginner school student.
	\item Translate the abstract into plain English. Replace technical words with everyday words
	\item Explain what each section of this paper contributes
	\item Explain the technical terms and equations simply.
	\item Explain how their experiment works and what they found
	\item Evaluate the paper’s methodology and reliability
	\item What new thing the paper as done or proposed and what was the earlier approach , aimed at a beginner school student.
  \end{itemize}

\item \textbf{Structure / Components Prompts/ Process \& Dataflow Prompts}
  \begin{itemize}
    \item List all tools / technologies / methods / methodologies used in this paper (e.g., models,libraries, datasets, metrics), aimed at a beginner school student
    \item What is the the initial input and final output of the work propesed by this paper. aimed at a beginner school student.
    \item What the pipeline stages which converts input(example a prompt) to final output(example mcq questions)
    \item Create a short checklist of what I’d need to implement a minimal working version (code libraries, hardware, dataset, hyperparameters).
	\item Explain step-by-step how input data is transformed into the output — write it as a numbered sequence.
	\item Make a table with 5 columns: stage name, input to this stage, description, tools used, outputs of this stage. your answer should be aimed at a beginner school stu- dent.write latex code. the table should not be truncated in page. all collumns should be visibel.
  \item Make a table: left column = stage name, middle = what happens there, right = inputs/outputs of that stage.
  \end{itemize}

\item \textbf{Visuals \& Diagrams Prompts}
  \begin{itemize}
    \item Create a concise flowchart description (text) I can use to draw a diagram: nodes and arrows labeled.
    \item Create a diagram/table/flowchart showing how all these components interact?
	\item Convert this into a visual diagram (flowchart-style image) showing arrows and boxes
	\item Give a step-by-step narrative explanation of each stage (like how data flows and decisions happen)? your answer should be aimed at a beginner school student
  \item Convert the pipeline into pseudocode (short, main functions only).
	\item Generate a flowchart description (boxes + arrows) for the model pipeline suitable for slide creation.
    \item Create a simple diagram caption and a short legend for each box in the flowchart.
    \item List the exact labels and a recommended layout (horizontal/vertical) to draw the diagram in PowerPoint.
  \end{itemize}

\item \textbf{Math, Algorithms, and Code Prompts}
  \begin{itemize}
    \item Explain the main equations line-by-line. For each variable, say what it represents and its units if any.
    \item Show a worked toy example (numerical) through the core equation / algorithm (use small numbers).
    \item Translate the method into Python-like pseudocode, with function names and comments.
    \item If I wanted a minimal reproducible code snippet, what are the 8–12 lines of code I could start with?
    \item Identify hyperparameters, their roles, and good starting values (with justification).
  \end{itemize}

\item \textbf{Evaluation, Baselines, and Experiments Prompts}
  \begin{itemize}
    \item List the datasets used, their sizes, and why they are appropriate or not.
    \item Summarize the evaluation metrics and explain what each one means in practice.
    \item Compare the paper’s method with the baselines: what improves and by how much (numbers)?
    \item Suggest 3 ablation experiments to test which parts of the model matter most.
  \end{itemize}

\item \textbf{Critique \& Limitations Prompts}
  \begin{itemize}
    \item List 8 strengths and 8 weaknesses of the paper — be concrete and specific.
    \item Identify any unstated or hidden assumptions the authors make.
    \item Find possible failure modes or data regimes where their approach will break.
    \item Point out questionable experimental choices or missing baselines.
  \end{itemize}

\item \textbf{Research Gap / Idea Generation Prompts}
  \begin{itemize}
    \item Based on this paper, suggest 6 follow-up research questions or directions.
    \item Propose 4 concrete, publishable project ideas that extend or improve this paper (include a one-sentence hypothesis and one experiment each).
    \item Find low-effort / high-impact experiments I could run in 1–2 weeks to get publishable results.
    \item Suggest related fields or communities (conferences/journals) that would be interested in this work.
	\item List research gaps or open problems from this paper
	\item Suggest related works or follow-up papers
	\item Based on these gaps, suggest possible research ideas for me
\end{itemize}

\item \textbf{Reproducibility \& Implementation Prompts}
  \begin{itemize}
    \item Give me a step-by-step reproducibility checklist: data download → preprocessing commands → training command → evaluation command.
    \item List exact hardware and runtime estimates (GPU type, memory, time) for training the model at published scale (rough estimate).
    \item Create a minimal dataset / toy dataset and the code to test the idea quickly.
  \end{itemize}

\item \textbf{Writing, Presentation \& Literature Mapping Prompts}
  \begin{itemize}
    \item Write a concise Related Work paragraph that places this paper among similar efforts (3–4 sentences).
    \item Draft a paragraph that summarizes this paper to include in my literature review.
    \item Create 6 slide titles and one-line bullet points for a 6-slide presentation of this paper.
    \item Find 8 papers cited by this paper and give a 1-sentence summary for each (if you can access them).
  \end{itemize}

\item \textbf{LLM-in-Education Specialization Prompts}
  \begin{itemize}
    \item Explain how this paper's method could be adapted for educational use-cases (MCQ generation, feedback, tutoring).
    \item Give 6 concrete ways a large language model could be integrated into this pipeline for education-specific improvements.
    \item List ethical / fairness / safety issues when using LLMs in education and how to test for them.
  \end{itemize}

\item \textbf{Practical Usage Prompts}
  \begin{itemize}
    \item I am uploading [paper.pdf]. Start by giving me: 1) TL;DR (1 sentence), 2) Beginner summary (3 bullets), 3) Top 5 weaknesses.
    \item Here is the PDF. Produce a slide deck outline (6 slides) and a diagram description of the pipeline.
    \item Take this paper and suggest 3 reproducible experiments I can run in 2 weeks on a single GPU; include datasets and approximate time.
  \end{itemize}

\end{enumerate}

	\section{Goal}
	
	\begin{enumerate}
		\item Learn how to read and understand research papers
		\item Extract the key ideas and identify research gaps
		\item Gradually build enough knowledge to design and publish your own paper in AI + Education (LLMs)
		\item LLm can generate
		\begin{enumerate}
			\item Plain-English summary (for beginner comprehension)
			\item Concept map (explaining how ideas are related)
			\item Methodology flow explanation
			\item Glossary of technical terms
			\item Critical evaluation (strengths, limitations, assumptions)
			\item Research gaps + possible extensions
			\item Template for your literature review
		\end{enumerate}
	\end{enumerate}

\end{document}
